\documentclass{beamer}
 
\mode<presentation> {

%\usetheme{Hannover}
\usetheme{Boadilla}

%\usetheme{AnnArbor}

\usecolortheme{whale}
%\usecolortheme{seagull}

%\usefonttheme{structuresmallcapsserif}

}

\usepackage[utf8]{inputenc}
 \usepackage{graphicx}
\usepackage{pdfpages}
\usepackage{array}
%\usepackage{animate}
\usepackage{amsmath}
\usepackage{multirow}
\usepackage{adjustbox}
\newcolumntype{C}[1]{>{\centering\let\newline\\\arraybackslash\hspace{0pt}}m{#1}}


\newcommand{\ebh}{\string~/bio.data/bio.lobster/figures/LFA3438Framework2019/} %change this to set figure directory
\newcommand{\ebhx}{\string~/bio.data/bio.lobster/figures/LFA3438Framework2019/figures/Brad/} %change this to set figure directory
\newcommand{\D}{.}

%Information to be included in the title page:
\title[LFA 34-38]{LFA 34-38 Framework 2019}
\author[Hubley, Cook, Howse and Denton]{Brad Hubley, Adam Cook, Victoria Howse, Cheryl Denton}
\institute[DFO]{Science Branch, Fisheries and Oceans Canada}
\date{Sept. 10-11, 2019}

 
 
\begin{document}
 
\frame{\titlepage}
 




\begin{frame}
\frametitle{Catch Per Unit Effort (CPUE)}
\begin{itemize}
\item Data comes from mandatory logs that begin in the mid 2000’s
\item Provid information on date, location (grid), effort, soak days and estimated catch.
\item Supplemented wherever possible with voluntary logs
\item Additional catch rate information was collected from historic studies
\end{itemize}
\end{frame}


\begin{frame}
\frametitle{Catch Per Unit Effort (CPUE)}
\begin{figure}
        \begin{center}
            \includegraphics[width=\textwidth,height=0.8\textheight,keepaspectratio]{\ebh CPUERawHistoric.pdf}
        \end{center}
    \end{figure}
\end{frame}



\begin{frame}
\frametitle{Catch Per Unit Effort (CPUE)}
\begin{itemize}
\item landings standardized by the number of trap hauls 
\item An indicator of Stock Status 
\item Assumed to be propotional to the exploited lobster population 
\item But also influenced by changing catchability 
\item Temperature is a major factor that affects catchability throughout the season 
\item Temperature needs to be assigned to individual log records in order to account for it's effect on catch rates 
\end{itemize}
\end{frame}



\begin{frame}
\frametitle{Temperature}
Temperature Model
\begin{itemize}
\item FSRS traps record temperature throughout the season and other data sources are available for other times and locations 
\item Generalized Additive Model (GAM) was used to predict temperature for a specific time and location
\item Continous time variable \textit{y}
\item Seasonal variability with Harmonics of the y variable, set up for an annual cycle (cos.y = cos(2$\pi$y), sin.y = sin(2$\pi$y))
\item Depth has a dampening effect on seasonal variability
\item Area as a factor
\end{itemize}
\end{frame}


\begin{frame}
\frametitle{Temperature}
Locations of Temperature Data
\begin{figure}
        \begin{center}
            \includegraphics[clip,trim={0 0 0 2cm},width=\textwidth,height=0.8\textheight,keepaspectratio]{\ebh TempDataMap.png}
        \end{center}
    \end{figure}
\end{frame}

\begin{frame}
\frametitle{Temperature}
Seasonal Temperature Pattern
\begin{figure}
        \begin{center}
            %\includegraphics[width=\textwidth,height=0.4\textheight,keepaspectratio]{\clam clam.jpg}
            \includegraphics[clip,trim={0 0.2cm 0 0.2cm },width=1\textwidth]{\ebh TempModel34.png}
        \end{center}
    \end{figure}
\begin{itemize}
\item Temperature assigned based on the date and depth of each log record
\item Depth and location are not reported in the log, so average depth of each grid was used
\end{itemize}
\end{frame}

\begin{frame}
\frametitle{Temperature}
Annual Temperature Trend
\begin{figure}
        \begin{center}
            \includegraphics[width=\textwidth,height=0.8\textheight,keepaspectratio]{\ebhx TempModelAnnual.png}
        \end{center}
    \end{figure}
\end{frame}


\begin{frame}
\frametitle{CPUE Model}
CPUE Model
\begin{itemize}
\item Modelled seperately for each area using a Generalized Linear Model (GLM)
\item Weight reported in each log record was log-transformed and offset by the log of the trap hauls
\item Predictors included day of season, predicted bottom temperature and year.
\item Day of season represents the depletion of the stock over the course of the season
\item Year was treated as a factor to allow free estimation of interannual variability
\item Random effects of vessel were not possible to estimate (too many vessels)
\item Different formulations of temperature and day of season were tested and the formulation with the lowest AIC included both temperature and day of season and their interaction 
\end{itemize}
\end{frame}



\begin{frame}
\frametitle{CPUE Model}
CPUE Model
\begin{itemize}
\item The GLM was used to predict CPUE on a particular day using the average temperature of the log records for the day
\item The annual index was the predicted CPUE on the first day of the season at the average temperature typically experienced on that day (across years)
\item Predicted daily CPUE starts off high in each LFA and drops exponentially as the season heads into the winter.
\item Predicted daily CPUE begins to increase again in the spring as the water temperatures begin to increase
\item In years where the temperature on the first day of the season was warmer than average the predicted daily CPUE starts higher than the annual index
\end{itemize}
\end{frame}




\begin{frame}
\frametitle{CPUE Model}
\begin{figure}
        \begin{center}
            \includegraphics[width=\textwidth,height=0.8\textheight,keepaspectratio]{\ebhx CPUEmodel1.png}
        \end{center}
    \end{figure}
\end{frame}








\end{document}


