\documentclass{beamer}
 
\mode<presentation> {

%\usetheme{Hannover}
\usetheme{Boadilla}

%\usetheme{AnnArbor}

\usecolortheme{whale}
%\usecolortheme{seagull}

%\usefonttheme{structuresmallcapsserif}

}

\usepackage[utf8]{inputenc}
 \usepackage{graphicx}
\usepackage{pdfpages}
\usepackage{array}
%\usepackage{animate}
\usepackage{amsmath}
\usepackage{multirow}
\usepackage{adjustbox}
\newcolumntype{C}[1]{>{\centering\let\newline\\\arraybackslash\hspace{0pt}}m{#1}}


\newcommand{\eb}{\string~/bio.data/bio.lobster/figures/ILTS/} %change this to set figure directory
\newcommand{\ebh}{\string~/bio.data/bio.lobster/figures/LFA3438Framework2019/} %change this to set figure directory
\newcommand{\ebhx}{\string~/bio.data/bio.lobster/figures/LFA3438Framework2019/figures/Brad/} %change this to set figure directory
\newcommand{\D}{.}

 
%Information to be included in the title page:
\title[ILTS Gear correction]{Inshore Lobster Trawl Survey: Gear Correction Factors}
\author[Hubley, Cook, Denton, Cassita-daRos]{Brad Hubley, Adam Cook, Cheryl Denton, and Manon Cassita-daRos}
\institute[DFO]{Science Branch, Fisheries and Oceans Canada}
\date{Framework Sept. 2019}

 
 
\begin{document}
 
\frame{\titlepage}
 

\begin{frame}
\frametitle{ITQ - ILTS}
History of the ITQ Survey
\begin{itemize}
    \item Groundfish ITQ survey mid-1990's
    \item Fixed station survey on set chosen within each 'grid' - Captain's choice
\end{itemize}
 
\begin{figure}
        \begin{center}
            \includegraphics[width=\textwidth,height=0.6\textheight,keepaspectratio]{\eb ITQSurveyBoxesFinal.jpg}
   \vspace{1cm}
    \end{center}
   
 \end{figure}

\end{frame}


\begin{frame}
\frametitle{ITQ - ILTS}
History of the ITQ Survey
\begin{itemize}
    \item Lobster catch increassed in the 2000's and began to be measured in 2005
    \item Became the Inshore Lobster Trawl Survey (ILTS) in 2013
\end{itemize}
 
\begin{figure}
        \begin{center}
            \includegraphics[width=\textwidth,height=0.6\textheight,keepaspectratio]{\eb ITQ.jpg}
   \vspace{1cm}
    \end{center}
   
 \end{figure}

\end{frame}



\begin{frame}
\frametitle{ILTS Survey}
History of the ILTS Survey
 
\begin{figure}
        \begin{center}
            2013
            \includegraphics[width=0.4\textwidth,height=0.38\textheight,keepaspectratio]{\eb ILTS2013.jpg}
            \includegraphics[width=0.4\textwidth,height=0.38\textheight,keepaspectratio]{\eb ILTS2014.jpg}
            2014\\
            2015
            \includegraphics[width=0.4\textwidth,height=0.38\textheight,keepaspectratio]{\eb ILTS2015.jpg}
            \includegraphics[width=0.4\textwidth,height=0.38\textheight,keepaspectratio]{\eb ILTS2016.jpg}
            2016
    \end{center}
   
 \end{figure}

\end{frame}


\begin{frame}
\frametitle{ILTS Survey}
Design and Swept area calculations
\begin{itemize}
\item Fixed Station Design 
\item A trawl mensuration system was used with Marport Sensors to record information on the wing spread, headline height and depth of the trawl
\item The start and end time for each tow is determined from a visual inspection of the depth data from the net sensors
\item A smoother is applied to GPS data recorded between the start and end times to calculate tow length 
\item Area is calculated using tow length and average wing spread
\end{itemize}
\end{frame}




\begin{frame}
\frametitle{ILTS Survey}
Depth profiles from Marport Sensors 
\begin{figure}
        \begin{center}
            \includegraphics[width=0.4\textwidth,height=0.4\textheight,keepaspectratio]{\eb Comparative2016/Depth92.png}
            \includegraphics[width=0.4\textwidth,height=0.4\textheight,keepaspectratio]{\eb Comparative2016/Depth114.png}\\
            \includegraphics[width=0.4\textwidth,height=0.4\textheight,keepaspectratio]{\eb Comparative2016/Depth48.png}
            \includegraphics[width=0.4\textwidth,height=0.4\textheight,keepaspectratio]{\eb Comparative2016/Depth81.png}
    \end{center}
   
 \end{figure}

\end{frame}


\begin{frame}
\frametitle{ILTS Survey}
\begin{figure}
        \begin{center}
            \includegraphics[clip,trim={0 0 0 0cm},width=\textwidth,height=0.8\textheight,keepaspectratio]{\eb Comparative2016/ILTS2016Tracks-2.png}
        \end{center}
    \end{figure}
\end{frame}



\begin{frame}
\frametitle{ILTS Comparative}
\begin{minipage}[c]{0.45\textwidth}
\begin{figure}
        \begin{center}
            \includegraphics[clip,trim={0 0 0 0cm},width=\textwidth,height=0.8\textheight,keepaspectratio]{\eb comparativeStations.png}
        \end{center}
    \end{figure}
\end{minipage}
\hfill
\begin{minipage}[c]{0.45\textwidth}
\begin{itemize}
    \item In 2016 Northeast Fisheries Science Center  Ecosystem Survey Trawl (NEST) was adopted
    \item 30 of 63 comparative stations  
\end{itemize}
\end{minipage}


\end{frame}


\begin{frame}
\frametitle{ILTS Comparative}
%Length Frequency
\begin{figure}
        \begin{center}
            NEST trawl
            \includegraphics[clip,trim={0 0 1 0cm},width=\textwidth,height=0.4\textheight,keepaspectratio]{\ebh CLFLobSurvLFA34NEST2016.png}\\
            Balloon trawl
            \includegraphics[clip,trim={0 0 1 0cm},width=\textwidth,height=0.4\textheight,keepaspectratio]{\ebh CLFLobSurvLFA34BALLOON2016.png}
        \end{center}
    \end{figure}
\end{frame}


\begin{frame}
\frametitle{ILTS Correction Factor}
Length based calibration coefficients
\begin{itemize}
\item Estimated using zero-inflated beta binomial (ZIBB) generalized additive models (GAMS) 
\item Allows the probabilities of the binomial model follow a beta distribution while accounting for the excess zeros
\item Data was filtered to size classes between 45 and 145mm 
%\item The conversion coefficient (ρ) is defined as μ / (1- μ + ν) where μ and ν are predicted for each size class between 45 and 145mm
\end{itemize}

\end{frame}



\begin{frame}
\frametitle{ILTS Correction Factor}
\begin{figure}
        \begin{center}
            \includegraphics[clip,trim={0 0 0 0cm},width=\textwidth,height=0.8\textheight,keepaspectratio]{\eb ZIBBresults.png}
        \end{center}
    \end{figure}
\end{frame}

\begin{frame}
\frametitle{ILTS Correction Factor}
\begin{figure}
        \begin{center}
            \includegraphics[clip,trim={0 0 0 0cm},width=\textwidth,height=0.8\textheight,keepaspectratio]{\eb ZIBBresultsCV.png}
        \end{center}
    \end{figure}
\end{frame}


\begin{frame}
\frametitle{ILTS Correction Factor}
Results
\begin{itemize}
\item Overall the NEST trawl captured more lobsters when the area swept was accounted for
\item Between lobsters of size 70mm to 120mm the conversion coefficient is about 3 and CVs are the lowest (0.13)
\item Over 120mm is similar (0.26) but higher CVs (0.5) 
\item Between 45mm to 70mm cofficients are higher (up to 8.3) and also more uncertain (CV up to 0.4)
\end{itemize}

\end{frame}




\end{document}


