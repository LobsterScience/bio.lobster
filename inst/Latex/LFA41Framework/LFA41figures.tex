\documentclass[11pt]{article}
\usepackage{graphicx}
\usepackage{subfig}
\usepackage{pdfcomment}
\usepackage{amsmath}
\usepackage{lscape}
\usepackage{hyperref}
\usepackage[top=2.4cm, bottom=2.4cm, left=3cm, right=3cm]{geometry}
\usepackage{fancyhdr}
\pagestyle{fancy}

\lhead{\bf Maritimes Region}
\rhead{\bf LFA41 - 2016}
\lfoot{October 11, 2016}
\cfoot{\thepage}
\renewcommand{\headrulewidth}{0.4pt}
\renewcommand{\footrulewidth}{0.4pt}
\newcommand{\D}{.}
\newcommand{\tl}{\textless}
\newcommand{\e}{/backup/bio_data/bio.lobster/figures/} %change this to set figure directory
\newcommand{\spm}{/backup/bio_data/bio.lobster/spmodelling/lfa41/}


\begin{document}

\begin{landscape}
% maps section
\begin{figure}
\centering
    \includegraphics[width=1\textwidth]{\e LFAMapATL.jpg}
    \caption{Map of the Lobster Fishing Areas in Atlantic Canada using the boundaries identified in the Atlantic fishery regulations.}

\end{figure}
\end{landscape}

\begin{figure}
\centering
    \includegraphics[width=0.8\textwidth]{\e oldoffshorebounds.jpg}
    \caption{Map showing the traditional offshore subareas used in pre 2009 assessments Crowell Basin, SW Browns, Georges Basin, SE Browns and Georges Bank.}

\end{figure}

\begin{figure}
\centering
    \includegraphics[width=.8\textwidth]{\e newoffshoreareas.jpg}
    \caption{Map showing the offshore zones used in current assessments. Zone 1 represents Crowell Basin, Zone 2 SW Browns, Zone 3 Georges Basin, Zone 4 Georges Bank and Zone 5 SE Browns. 
}

\end{figure}

\begin{landscape}
%\begin{figure}
%\centering
%\subfloat{\includegraphics[width=0.5\textwidth]{\e lfa41Logs\D 2002\D 2007\D .pdf}}
%\subfloat{\includegraphics[width=0.5\textwidth]{\e lfa41Logs\D 2007\D 2012\D .pdf}}\\
%\subfloat{\includegraphics[width=0.5\textwidth]{\e lfa41Logs\D 2013\D 2018\D .pdf}}
%\caption{Map showing spatial location of fishing in offshore LFA41 between 2003 to 2007 (topleft), 2008 to 2012 (topright) and 2013 to 2015 (bottom) as obtained from fishery %logbook information. Overlaid in red are the offshore polygons which will be used to describe the fishery data into zones. }
%\end{figure}
%\clearpage

\begin{figure}
\centering
\subfloat{\includegraphics[width=0.65\textwidth]{\e summerstratamap.pdf}}
\subfloat{\includegraphics[width=0.5\textwidth]{\e summerstrata41closeup.pdf}}\\
\caption{ Map of Lobster Fishing Areas (LFAs) in black overlain with the full DFO Summer RV survey strata shown in red (left). Close-up of the fished areas of Lobster Fishing Area 41 (blue line) with the DFO Summer RV survey strata included in survey trends outlined in red (right).}
\end{figure}

\end{landscape}
\begin{figure}

    \includegraphics[width=1\textwidth]{\e georgesmap41.pdf}
    \caption{Currently fished areas of Lobster Fishing Area 41 (blue line) with the DFO Georges Bank Spring strata from their depth stratified survey shown in red and green. The strata outlined in green are those used in survey trends from the Georges Bank Survey.}

\end{figure}

\begin{landscape}
\begin{figure}
\centering
\subfloat{\includegraphics[width=.8\textwidth]{\e americanmap41full.pdf}}
\subfloat{\includegraphics[width=0.8\textwidth]{\e americanmap41.pdf}}\\
\caption{ Currently fished areas of Lobster Fishing Area 41 (blue line) with the NEFSC spring and autumn strata from their depth stratified survey shown in red (left). Currently fished area of Lobster Fishing Area 41 (blue line) with the NEFSC spring and autumn strata (shown in red) used for the analysis of survey trends (right).}
\end{figure}


\end{landscape}

%survey length frequencies 
\begin{figure}
\centering
    \includegraphics[width=0.8\textwidth]{\e DFORVSurveyLengthFreqAllv41.png}
    \caption{Comparison of sampled length frequencies from the DFO summer RV survey for the entire surveyed area (red) and the lobsters sampled within LFA14 (black). Densities were scaled to the maximum density estimated.}
\end{figure}

%survey bubbles
        \begin{figure}
        \centering
        \subfloat{\includegraphics[clip,trim={0 2.1cm 0.3cm 2.1cm},width=0.37\textwidth]{\e surveyBubblesDFOSummer\D 1970\D 1974.pdf}}
        \subfloat{\includegraphics[clip,trim={0 2.1cm 0.3cm 2.1cm},width=0.37\textwidth]{\e surveyBubblesDFOSummer\D 1975\D 1979.pdf}}
        \subfloat{\includegraphics[clip,trim={0 2.1cm 0.3cm 2.1cm},width=0.37\textwidth]{\e surveyBubblesDFOSummer\D 1980\D 1984.pdf}}\\
        \subfloat{\includegraphics[clip,trim={0 2.1cm 0.3cm 2.1cm},width=0.37\textwidth]{\e surveyBubblesDFOSummer\D 1985\D 1989.pdf}}
        \subfloat{\includegraphics[clip,trim={0 2.1cm 0.3cm 2.1cm},width=0.37\textwidth]{\e surveyBubblesDFOSummer\D 1990\D 1994.pdf}}
        \subfloat{\includegraphics[clip,trim={0 2.1cm 0.3cm 2.1cm},width=0.37\textwidth]{\e surveyBubblesDFOSummer\D 1995\D 1999.pdf}}\\
        \subfloat{\includegraphics[clip,trim={0 2.1cm 0.3cm 2.1cm},width=0.37\textwidth]{\e surveyBubblesDFOSummer\D 2000\D 2004.pdf}}
        \subfloat{\includegraphics[clip,trim={0 2.1cm 0.3cm 2.1cm},width=0.37\textwidth]{\e surveyBubblesDFOSummer\D 2005\D 2009.pdf}}
        \subfloat{\includegraphics[clip,trim={0 2.1cm 0.3cm 2.1cm},width=0.37\textwidth]{\e surveyBubblesDFOSummer\D 2010\D 2014.pdf}}\\
        \subfloat{\includegraphics[clip,trim={0 2.1cm 0.3cm 2.1cm},width=0.37\textwidth]{\e surveyBubblesDFOSummer\D 2015\D 2015.pdf}}\\


         \caption{Map of the abundance of lobster captured during DFO's summer RV survey of the Scotian Shelf. Strata boundaries are outlined in red and LFA 41 stock boundaries are outlined in blue. Size of the symbols are scaled to the number observed within each tow.}
        \end{figure}
        \clearpage



        \begin{figure}
        \centering
        \subfloat{\includegraphics[clip,trim={0 2.1cm 0.3cm 2.1cm},width=0.37\textwidth]{\e surveyBubblesNEFSCSpring\D 1969\D 1974.pdf}}
        \subfloat{\includegraphics[clip,trim={0 2.1cm 0.3cm 2.1cm},width=0.37\textwidth]{\e surveyBubblesNEFSCSpring\D 1975\D 1979.pdf}}
        \subfloat{\includegraphics[clip,trim={0 2.1cm 0.3cm 2.1cm},width=0.37\textwidth]{\e surveyBubblesNEFSCSpring\D 1980\D 1984.pdf}}\\
        \subfloat{\includegraphics[clip,trim={0 2.1cm 0.3cm 2.1cm},width=0.37\textwidth]{\e surveyBubblesNEFSCSpring\D 1985\D 1989.pdf}}
        \subfloat{\includegraphics[clip,trim={0 2.1cm 0.3cm 2.1cm},width=0.37\textwidth]{\e surveyBubblesNEFSCSpring\D 1990\D 1994.pdf}}
        \subfloat{\includegraphics[clip,trim={0 2.1cm 0.3cm 2.1cm},width=0.37\textwidth]{\e surveyBubblesNEFSCSpring\D 1995\D 1999.pdf}}\\
        \subfloat{\includegraphics[clip,trim={0 2.1cm 0.3cm 2.1cm},width=0.37\textwidth]{\e surveyBubblesNEFSCSpring\D 2000\D 2004.pdf}}
        \subfloat{\includegraphics[clip,trim={0 2.1cm 0.3cm 2.1cm},width=0.37\textwidth]{\e surveyBubblesNEFSCSpring\D 2005\D 2009.pdf}}
        \subfloat{\includegraphics[clip,trim={0 2.1cm 0.3cm 2.1cm},width=0.37\textwidth]{\e surveyBubblesNEFSCSpring\D 2010\D 2014.pdf}}\\
        \subfloat{\includegraphics[clip,trim={0 2.1cm 0.3cm 2.1cm},width=0.37\textwidth]{\e surveyBubblesNEFSCSpring\D 2015\D 2015.pdf}}\\


         \caption{Map of the abundance of lobster captured during during NEFSC's Spring Survey of the Gulf of Maine, Georges Bank and Scotian Shelf. Strata boundaries are outlined in red and LFA 41 stock boundaries are outlined in blue. Size of the symbols are scaled to the number observed within each tow.}
        \end{figure}
        \clearpage


        \begin{figure}
        \centering
        \subfloat{\includegraphics[clip,trim={0 2.1cm 0.3cm 2.1cm},width=0.37\textwidth]{\e surveyBubblesNEFSCFall\D 1969\D 1974.pdf}}
        \subfloat{\includegraphics[clip,trim={0 2.1cm 0.3cm 2.1cm},width=0.37\textwidth]{\e surveyBubblesNEFSCFall\D 1975\D 1979.pdf}}
        \subfloat{\includegraphics[clip,trim={0 2.1cm 0.3cm 2.1cm},width=0.37\textwidth]{\e surveyBubblesNEFSCFall\D 1980\D 1984.pdf}}\\
        \subfloat{\includegraphics[clip,trim={0 2.1cm 0.3cm 2.1cm},width=0.37\textwidth]{\e surveyBubblesNEFSCFall\D 1985\D 1989.pdf}}
        \subfloat{\includegraphics[clip,trim={0 2.1cm 0.3cm 2.1cm},width=0.37\textwidth]{\e surveyBubblesNEFSCFall\D 1990\D 1994.pdf}}
        \subfloat{\includegraphics[clip,trim={0 2.1cm 0.3cm 2.1cm},width=0.37\textwidth]{\e surveyBubblesNEFSCFall\D 1995\D 1999.pdf}}\\
        \subfloat{\includegraphics[clip,trim={0 2.1cm 0.3cm 2.1cm},width=0.37\textwidth]{\e surveyBubblesNEFSCFall\D 2000\D 2004.pdf}}
        \subfloat{\includegraphics[clip,trim={0 2.1cm 0.3cm 2.1cm},width=0.37\textwidth]{\e surveyBubblesNEFSCFall\D 2005\D 2009.pdf}}
        \subfloat{\includegraphics[clip,trim={0 2.1cm 0.3cm 2.1cm},width=0.37\textwidth]{\e surveyBubblesNEFSCFall\D 2010\D 2014.pdf}}\\
        \subfloat{\includegraphics[clip,trim={0 2.1cm 0.3cm 2.1cm},width=0.37\textwidth]{\e surveyBubblesNEFSCFall\D 2015\D 2015.pdf}}\\


         \caption{Map of the abundance of lobster captured during NEFSC's Fall Survey of the Gulf of Maine, Georges Bank and Scotian Shelf. Strata boundaries are outlined in red and LFA 41 stock boundaries are outlined in blue. Size of the symbols are scaled to the number observed within each tow.}
        \end{figure}
        \clearpage



        \begin{figure}
        \centering
        \subfloat{\includegraphics[clip,trim={0 2.1cm 0.3cm 2.1cm},width=0.37\textwidth]{\e surveyBubblesDFOGeorges\D 1987\D 1994.pdf}}
        \subfloat{\includegraphics[clip,trim={0 2.1cm 0.3cm 2.1cm},width=0.37\textwidth]{\e surveyBubblesDFOGeorges\D 1995\D 1999.pdf}}
        \subfloat{\includegraphics[clip,trim={0 2.1cm 0.3cm 2.1cm},width=0.37\textwidth]{\e surveyBubblesDFOGeorges\D 2000\D 2004.pdf}}\\
        \subfloat{\includegraphics[clip,trim={0 2.1cm 0.3cm 2.1cm},width=0.37\textwidth]{\e surveyBubblesDFOGeorges\D 2005\D 2009.pdf}}
        \subfloat{\includegraphics[clip,trim={0 2.1cm 0.3cm 2.1cm},width=0.37\textwidth]{\e surveyBubblesDFOGeorges\D 2010\D 2014.pdf}}
        \subfloat{\includegraphics[clip,trim={0 2.1cm 0.3cm 2.1cm},width=0.37\textwidth]{\e surveyBubblesDFOGeorges\D 2015\D 2015.pdf}}\\

         \caption{Map of the abundance of lobster captured during DFO's Georges Bank Survey. Strata boundaries are outlined in red and LFA 41 stock boundaries are outlined in blue. Size of the symbols are scaled to the number observed within each tow.}
        \end{figure}
        \clearpage


%survey efficiency section


        \begin{figure}
        \centering
        \subfloat{\includegraphics[width=0.5\textwidth]{\e lfa41DFObase.pdf}}
        \subfloat{\includegraphics[width=0.5\textwidth]{\e lfa41DFOrestratified.pdf}}\\
        \subfloat{\includegraphics[width=0.5\textwidth]{\e lfa41DFOadjrestratified.pdf}}
        \caption{Survey efficiency of DFO RV base survey (topleft), DFO RV restratified survey (topright) and DFO RV survey restratified to adjacent areas (bottom) from 1999 to 2015. Percent efficiency refers to changes in either strata or allocation scheme relative to a simple random survey }
        \end{figure}
        \clearpage



        \begin{figure}
        \centering
        \subfloat{\includegraphics[width=0.5\textwidth]{\e lfa41NEFSCspringbase.pdf}}
        \subfloat{\includegraphics[width=0.5\textwidth]{\e lfa41NEFSCspringrestratified.pdf}}\\
        \subfloat{\includegraphics[width=0.5\textwidth]{\e lfa41NEFSCspringrestratifiedadjacent.pdf}}
        \caption{Survey efficiency of NEFSC Spring base survey (topleft), NEFSC Spring restratified survey (topright) and NEFSC Spring survey restratified to adjacent areas (bottom) from 1999 to 2015. Percent efficiency refers to changes in either strata or allocation scheme relative to a simple random survey }
        \end{figure}
        \clearpage


        \begin{figure}
        \centering
        \subfloat{\includegraphics[width=0.5\textwidth]{\e lfa41NEFSCfallbase.pdf}}
        \subfloat{\includegraphics[width=0.5\textwidth]{\e lfa41NEFSCfalrestratified.pdf}}\\
        \subfloat{\includegraphics[width=0.5\textwidth]{\e lfa41NEFSCFallrestratifiedadjacent.pdf}}
        \caption{Survey efficiency of NEFSC Fall base survey (topleft), NEFSC Fall restratified survey (topright) and NEFSC Fall survey restratified to adjacent areas (bottom) from 1999 to 2015. Percent efficiency refers to changes in either strata or allocation scheme relative to a simple random survey }
        \end{figure}
        \clearpage



        \begin{figure}
        \centering
        \subfloat{\includegraphics[width=0.85\textwidth]{\e lfa41georgesefficiency.pdf}}
        \caption{Survey efficiency of Georges Bank survey from 1999 to 2015. Percent efficiency refers to changes in either strata or allocation scheme relative to a simple random survey }
        \end{figure}
        \clearpage



%%%%%%%%%%%%%
%Biomass dynamic modelling figures
%%%%%%%%%%%%%
\begin{figure}
\centering
        \subfloat{\includegraphics[width=0.85\textwidth]{\spm SurveyIndicesModeledB.png}}
        \caption{Plot of biomass dynamic model fits (black line) along with DFO Summer RV survey commercial (red), NEFSC Spring survey commercial biomass (purple), NEFSC Autumn survey commercial biomass (green) and DFO Georges Bank survey commercial biomass (blue). Each survey index was adjusted by their specific modeled estimate of \emph{q} to match the scale of the modelled biomass.}


\end{figure}

\begin{landscape}
\begin{figure}
\centering
        \subfloat{\includegraphics[width=0.65\textwidth]{\spm biomass\D timeseriesfourI.png}}
        \subfloat{\includegraphics[width=0.65\textwidth]{\spm fishingmortality\D timeseries\D fourI.png}}
       \caption{Time series of fishable biomass from  biomass dynamic model fits (left) and related time series of fishing mortality rates (right). In each plot the density distribution of the posterior fishable biomass or fishing mortality estimates are presented in grey with the darkest areas representing medians and the lightest representing the 50\% credible intervals.}

\end{figure}
     \clearpage




\begin{figure}
\centering
        \subfloat{\includegraphics[clip,trim={0 1.cm 0.3cm 2.1cm},width=0.37\textwidth]{\spm priorposteriorr.png}}
        \subfloat{\includegraphics[clip,trim={0 1.cm 0.3cm 2.1cm},width=0.37\textwidth]{\spm priorposteriorK.png}}
        \subfloat{\includegraphics[clip,trim={0 1.cm 0.3cm 2.1cm},width=0.37\textwidth]{\spm priorposteriorq1.png}}\\
        \subfloat{\includegraphics[clip,trim={0 1.cm 0.3cm 2.1cm},width=0.37\textwidth]{\spm priorposteriorq2.png}}
        \subfloat{\includegraphics[clip,trim={0 1.cm 0.3cm 2.1cm},width=0.37\textwidth]{\spm priorposteriorq3.png}}
        \subfloat{\includegraphics[clip,trim={0 1.cm 0.3cm 2.1cm},width=0.37\textwidth]{\spm priorposteriorq4.png}}\\
        \caption{Prior (red lines) and posterior distributions (bars) from the biomass dynamic modelling of from LFA41 lobster stock. Top row figures from left to right represent the intrinsic growth parameter \emph{r}, carrying capacity \emph{K} and the DFO Summer RV survey proportionality constant \emph{q}. Bottom row figures from left to right represent \emph{q} for the NEFSC spring survey, \emph{q} for the NEFSC Autumn survey and \emph{q} for the DFO Georges Bank survey. }
        
\end{figure}
     \clearpage


\begin{figure}
\centering
        \subfloat{\includegraphics[clip,trim={0 1cm 0.3cm 2.1cm},width=0.37\textwidth]{\spm priorposteriorsdp.png}}
        \subfloat{\includegraphics[clip,trim={0 1cm 0.3cm 2.1cm},width=0.37\textwidth]{\spm priorposteriorsd\D oa.png}}
        \subfloat{\includegraphics[clip,trim={0 1cm 0.3cm 2.1cm},width=0.37\textwidth]{\spm priorposteriorsd\D ob.png}}\\
        \subfloat{\includegraphics[clip,trim={0 1cm 0.3cm 2.1cm},width=0.37\textwidth]{\spm priorposteriorsd\D oc.png}}
        \subfloat{\includegraphics[clip,trim={0 1cm 0.3cm 2.1cm},width=0.37\textwidth]{\spm priorposteriorsd\D od.png}}
         \caption{Prior (red lines) and posterior distributions (bars) from the biomass dynamic modelling of from LFA41 lobster stock. Top row figures from left to right represent the process error $\tau$, the observation error $\sigma$ associated with DFO Summer RV survey and the observation error $\sigma$ associated with NEFSC Spring survey. Bottom row figures from left to right represent observation error $\sigma$ for the NEFSC Autumn survey and the observation error $\sigma$ for the DFO Georges Bank survey. }
       
        
\end{figure}
\end{landscape}
     \clearpage

\begin{figure}
\centering
\includegraphics[width=0.85\textwidth]{\spm CarringCapacityPriorBiomasstoKratio.png}
\caption{Impact of changing the prior mean for carrying capacity, \emph{K} on modelled biomass trends using the biomass dynamic model for LFA41 lobster. The lines represent the time series of the ratio of the time series of modelled median biomass to the median of modelled \emph{K} using increasing means for prior distribution on \emph{K}. }

\end{figure}
     \clearpage

\begin{landscape}
\begin{figure}
\centering
 \subfloat{\includegraphics[clip,trim={0 0.7cm 0.3cm 1.56cm},width=0.33\textwidth]{\spm HCRKsens69.png}}
 \subfloat{\includegraphics[clip,trim={0 0.7cm 0.3cm 1.56cm},width=0.33\textwidth]{\spm HCRKsens104.png}}
 \subfloat{\includegraphics[clip,trim={0 0.7cm 0.3cm 1.56cm},width=0.33\textwidth]{\spm HCRKsens139.png}}\\
 \subfloat{\includegraphics[clip,trim={0 0.7cm 0.3cm 1.56cm},width=0.33\textwidth]{\spm HCRKsens173.png}}
 \subfloat{\includegraphics[clip,trim={0 0.7cm 0.3cm 1.56cm},width=0.33\textwidth]{\spm HCRKsens208.png}}
 \subfloat{\includegraphics[clip,trim={0 0.7cm 0.3cm 1.56cm},width=0.33\textwidth]{\spm HCRKsens243.png}}\\
 \subfloat{\includegraphics[clip,trim={0 0.7cm 0.3cm 1.56cm},width=0.33\textwidth]{\spm HCRKsens277.png}}
 \subfloat{\includegraphics[clip,trim={0 0.7cm 0.3cm 1.56cm},width=0.33\textwidth]{\spm HCRKsens312.png}}
 \subfloat{\includegraphics[clip,trim={0 0.7cm 0.3cm 1.56cm},width=0.33\textwidth]{\spm HCRKsens347.png}}



\caption{Phase plot showing the impact of changing the prior mean for carrying capacity, \emph{K} on modelled biomass trends in relation to reference points. Each plot represents the estimated median biomass and reference points determined from biomass dynamic model parameters for a model run using a different mean prior on \emph{K}. The mean of the prior for each model run was shown in each figure title.}

\end{figure}
\end{landscape}
     \clearpage


%Biomass and Fishery reference points

%DFO Summer 
\begin{landscape}
\begin{figure}
\centering
         \subfloat{\includegraphics[width=0.65\textwidth]{\e BCPDFORefpointsNewArea.png}}
        \subfloat{\includegraphics[width=0.7\textwidth]{\e DFORefpointsNewArea.png}}
       \caption{Commercial biomasses (kt) for the DFO summer RV (RV41) survey. (Left) Results from bayesian change point analysis to determine the probability of a change in commercial biomasses as an indicator of changing productivity regime. Upper panel represent the posterior means of the bayesian change point model (red line) along with the input values for log(commercial biomass). Lower panel represents the probability of a change point occurring at specific time (black line). (Right) Commercial biomass time series along with the running median (red line) the median of the five lowest non zero biomasses (\emph{proposed LRP}; orange) and the medians for the full time series (USR; purple), the lower productivity period (1970-1999; blue; LRP) and 40\% of the median of the higher productivity period (2000-2015; green; \emph{proposed USR}). }

\end{figure}
\end{landscape}

\begin{figure}
\centering
        \subfloat{\includegraphics[width=0.7\textwidth]{\e relFDFOSurvey.png}}
       \caption{Relative fishing mortality (relF) for the DFO Summer RV (RV41) survey and landings. Results from bayesian change point analysis from commercial biomass trends were used to inform a change in productivity regime and hence relative F. Relative F time series along with the running median (red line) and the medians for the full time series (purple) and the lower productivity period (1981 - 1999;blue; \emph{proposed RR}). }
\end{figure}
     \clearpage


\begin{landscape}
\begin{figure}
\centering
         \subfloat{\includegraphics[width=0.45\textwidth]{\e HCRLongTermDataDFOSurvey.png}}
        \subfloat{\includegraphics[width=0.45\textwidth]{\e HCRDataDFOSurvey.png}}\\
        \subfloat{\includegraphics[width=0.45\textwidth]{\e HCRllbLongTermDataDFOSurvey.png}}
        \subfloat{\includegraphics[width=0.45\textwidth]{\e HCRllbDataDFOSurvey.png}}\\
       
       \caption{Phase plots showing the impact of different choices of reference points on the stock status zones for offshore American Lobster LFA41 using DFO Summer RV survey (RV41) commercial biomass (kt) and relative F. Left panels  - USR and RR were defined using the medians of the entire time series of biomass or relative fishing mortality. Right panels - USR and RR were defined using the medians of commercial biomass and relative F for the upper (2000 - 2015) and lower (1981 - 1999) productivity periods respectively. In upper panels LRP was defined as the median biomass during the lower productivity period. In lower panels LRP was defined as the median of the five lowest non zero biomasses. Time series trends of fishable biomass and fishing mortality were represented by the three year running medians. }

\end{figure}
\end{landscape}

%###### Refernce points spring

\begin{landscape}
\begin{figure}
\centering
         \subfloat{\includegraphics[width=0.65\textwidth]{\e BCPSpringRefpointsNewArea.png}}
        \subfloat{\includegraphics[width=0.7\textwidth]{\e SpringRefpointsNewArea.png}}
       \caption{Commercial biomasses (kt) for the NEFSC Spring (NSpr41) survey. (Left) Results from bayesian change point analysis to determine the probability of a change in commercial biomasses as an indicator of changing productivity regime. Upper panel represent the posterior means of the bayesian change point model (red line) along with the input values for log(commercial biomass). Lower panel represents the probability of a change point occurring at specific time (black line). (Right) Commercial biomass time series along with the running median (red line), the median of the five lowest non zero biomasses (\emph{proposed LRP}; orange) and the medians for the full time series (USR; purple), the lower productivity period (1969-2001; blue; LRP) and 40\% of the median of the higher productivity period (2002-2015; green; \emph{proposed USR}). }

\end{figure}
\end{landscape}
     \clearpage


\begin{figure}
\centering
        \subfloat{\includegraphics[width=0.7\textwidth]{\e relFSpringSurvey.png}}
       \caption{Relative fishing mortality (relF) for the NEFSC Spring (NSpr41) survey and landings. Results from bayesian change point analysis from commercial biomass trends were used to inform a change in productivity regime and hence relative F. Relative F time series along with the running median (red line) and the medians for the full time series (purple) and the lower productivity period (1981 - 2001;blue; \emph{proposed RR}). }
\end{figure}


\begin{landscape}
\begin{figure}
\centering
         \subfloat{\includegraphics[width=0.45\textwidth]{\e HCRLongTermDataSpringSurvey.png}}
        \subfloat{\includegraphics[width=0.45\textwidth]{\e HCRDataSpringSurvey.png}}\\
                 \subfloat{\includegraphics[width=0.45\textwidth]{\e HCRllbLongTermDataSpringSurvey.png}}
        \subfloat{\includegraphics[width=0.45\textwidth]{\e HCRllbDataSpringSurvey.png}}

      \caption{Phase plots showing the impact of different choices of reference points on the stock status zones for offshore American Lobster LFA41 using NEFSC Spring survey (NSpr41) commercial biomass (kt) and relative F. Left panels  - USR and RR were defined using the medians of the entire time series of biomass or relative fishing mortality. Right panels - USR and RR were defined using the medians of commercial biomass and relative F for the upper (2002 - 2015) and lower (1981 - 2001) productivity periods respectively. In upper panels LRP was defined as the median biomass during the lower productivity period. In lower panels LRP was defined as the median of the five lowest non zero biomasses in the time series. Time series trends of fishable biomass and fishing mortality were represented by the three year running medians. }

\end{figure}
\end{landscape}

%###### Refernce points autumn



\begin{landscape}
\begin{figure}
\centering
         \subfloat{\includegraphics[width=0.65\textwidth]{\e BCPAutumnRefpointsNewArea.png}}
        \subfloat{\includegraphics[width=0.7\textwidth]{\e AutumnRefpointsNewArea.png}}
       \caption{Commercial biomasses (kt) for the NEFSC Autumn (NAut41) survey. (Left) Results from bayesian change point analysis to determine the probability of a change in commercial biomasses as an indicator of changing productivity regime. Upper panel represent the posterior means of the bayesian change point model (red line) along with the input values for log(commercial biomass). Lower panel represents the probability of a change point occurring at specific time (black line). (Right) Commercial biomass time series along with the running median (red line), the median of the five lowest non zero biomasses (\emph{proposed LRP}; orange) and the medians for the full time series (USR; purple), the lower productivity period (1969-2000; blue; LRP) and 40\% of the median of the higher productivity period (2000-2015; green; \emph{proposed USR}). }

\end{figure}
\end{landscape}


\begin{figure}
\centering
        \subfloat{\includegraphics[width=0.7\textwidth]{\e relFAutumnSurvey.png}}
       \caption{Relative fishing mortality (relF) for the NEFSC Autumn (NAut41) survey and landings. Results from bayesian change point analysis from commercial biomass trends were used to inform a change in productivity regime and hence relative F. Relative F time series along with the running median (red line) and the medians for the full time series (purple) and the lower productivity period (1981 - 2000;blue; \emph{proposed RR}). }
\end{figure}


\begin{landscape}
\begin{figure}
\centering
         \subfloat{\includegraphics[width=0.45\textwidth]{\e HCRLongTermDataAutumnSurvey.png}}
        \subfloat{\includegraphics[width=0.45\textwidth]{\e HCRDataAutumnSurvey.png}}\\
        \subfloat{\includegraphics[width=0.45\textwidth]{\e HCRllbLongTermDataAutumnSurvey.png}}
        \subfloat{\includegraphics[width=0.45\textwidth]{\e HCRllbDataAutumnSurvey.png}}
      

      \caption{Phase plots showing the impact of different choices of reference points on the stock status zones for offshore American Lobster LFA41 using NEFSC Autumn survey (NAut41) commercial biomass (kt) and relative F. Left panels  - USR and RR were defined using the medians of the entire time series of biomass or relative fishing mortality. Right panels - USR and RR were defined using the medians of commercial biomass and relative F for the upper (2001 - 2015) and lower (1981 - 2000) productivity periods respectively. In upper panels LRP was defined as the median biomass during the lower productivity period. In lower panels LRP was defined as the median of the five lowest non zero biomasses in the time series. Time series trends of fishable biomass and fishing mortality were represented by the three year running medians. }

\end{figure}
\end{landscape}


%############### Referece points Georges
\begin{landscape}
\begin{figure}
\centering
         \subfloat{\includegraphics[width=0.65\textwidth]{\e BCPGeorgesRefpointsNewArea.png}}
        \subfloat{\includegraphics[width=0.7\textwidth]{\e GeorgesRefpointsNewArea.png}}
       \caption{Commercial biomasses (kt) for the DFO Georges Bank (GB) survey. (Left) Results from bayesian change point analysis to determine the probability of a change in commercial biomasses as an indicator of changing productivity regime. Upper panel represent the posterior means of the bayesian change point model (red line) along with the input values for log(commercial biomass). Lower panel represents the probability of a change point occurring at specific time (black line). (Right) Commercial biomass time series along with the running median (red line), the median of the five lowest non zero biomasses (\emph{proposed LRP}; orange) and the medians for the full time series (USR; purple), the lower productivity period (1987-1999; blue; LRP) and 40\% of the median of the higher productivity period (2000-2015; green; \emph{proposed USR}). }

\end{figure}
\end{landscape}
     \clearpage

\begin{figure}
\centering
        \subfloat{\includegraphics[width=0.7\textwidth]{\e relFGeorgesSurvey.png}}
       \caption{Relative fishing mortality (relF) for the DFO Georges (GB) survey and landings. Results from bayesian change point analysis from commercial biomass trends were used to inform a change in productivity regime and hence relative F. Relative F time series along with the running median (red line) and the medians for the full time series (purple) and the lower productivity period (1987 - 1999;blue; \emph{proposed RR}). }
\end{figure}
     \clearpage


\begin{landscape}
\begin{figure}
\centering
         \subfloat{\includegraphics[width=0.45\textwidth]{\e HCRLongTermDataGeorgesSurvey.png}}
        \subfloat{\includegraphics[width=0.45\textwidth]{\e HCRDataGeorgesSurvey.png}}\\
           \subfloat{\includegraphics[width=0.45\textwidth]{\e HCRllbLongTermDataGeorgesSurvey.png}}
        \subfloat{\includegraphics[width=0.45\textwidth]{\e HCRllbDataGeorgesSurvey.png}}
     
      \caption{Phase plots showing the impact of different choices of reference points on the stock status zones for offshore American Lobster LFA41 using DFO Georges Bank survey (GB) commercial biomass (kt) and relative F. Left panels  - USR and RR were defined using the medians of the entire time series of biomass or relative fishing mortality. Right panels - USR and RR were defined using the medians of commercial biomass and relative F for the upper (2000 - 2015) and lower (1987 - 1999) productivity periods respectively. In upper panels LRP was defined as the median biomass during the lower productivity period. In lower panels LRP was defined as the median of the five lowest non zero biomasses in the time series. Time series trends of fishable biomass and fishing mortality were represented by the three year running medians. }


\end{figure}
\end{landscape}
     \clearpage

%Reproducdtive potential refernce points
%DFO
\begin{figure}

\centering
    \subfloat{\includegraphics[width=0.7\textwidth]{\e RefsRepPotDFO.png}}
\caption{Reproductive potential in millions of eggs estimated from DFO Summer RV (RV41) survey American lobster population weighted fecundity estimates. Red line represents a three year running median. Green line represents the upper boundary estimated as 40\% of the median of 2000 - 2015. }
\end{figure}

%NEFSC spring
\begin{landscape}
\begin{figure}
\centering
        \subfloat{\includegraphics[width=0.65\textwidth]{\e BCPRepPotNEFSCSpring.png}}
        \subfloat{\includegraphics[width=0.7\textwidth]{\e RefsRepPotNEFSCSpring.png}}
    
\caption{Reproductive potential in millions of eggs estimated from NEFSC Spring survey (NSpr41) American lobster population weighted fecundity.  (Left) Bayesian change point analysis to determine if a shift in productivity regime was evident. Upper panel represents the posterior means of the bayesian change point model (red line) along with the input values for log(Reproductive potential). Lower panel represents the probability of a change point occurring at specific time (black line). (Right) Reproductive potential time series along with the running median (red line), the median of the five lowest non zero biomasses (lower boundary; orange) and the medians for the full time series (upper boundary; purple), the lower productivity period (1969-2001; blue; lower boundary) and 40\% of the median of the higher productivity period (2002-2015; green; proposed upper boundary). }
\end{figure}
\end{landscape}


%NEFSC autumn
\begin{landscape}
\begin{figure}
\centering
        \subfloat{\includegraphics[width=0.65\textwidth]{\e BCPRepPotNEFSCAutumn.png}}
        \subfloat{\includegraphics[width=0.7\textwidth]{\e RefsRepPotNEFSCAutumn.png}}
    
\caption{Reproductive potential in millions of eggs estimated from NEFSC Autumn survey (NAut41) American lobster population weighted fecundity.  (Left) Bayesian change point analysis to determine if a shift in productivity regime was evident. Upper panel represents the posterior means of the bayesian change point model (red line) along with the input values for log(Reproductive potential). Lower panel represents the probability of a change point occurring at specific time (black line). (Right) Reproductive potential time series along with the running median (red line), the median of the five lowest non zero biomasses (lower boundary; orange) and the medians for the full time series (upper boundary; purple), the lower productivity period (1969-2000; blue; lower boundary) and 40\% of the median of the higher productivity period (2001-2015; green; proposed upper boundary). }
\end{figure}
\end{landscape}

%DFO
\begin{figure}

\centering
    \subfloat{\includegraphics[width=0.7\textwidth]{\e RefsRepPotGeorges.png}}
\caption{Reproductive potential in millions of eggs estimated from DFO Georges Bank RV (GB) survey American lobster population weighted fecundity estimates. Red line represents a three year running median. }
\end{figure}



%%%%%%%%%%%%%%%relative F and BCP plots


%\begin{figure}
%\centering
%        \subfloat{\includegraphics[width=0.7\textwidth]{\e relFDFOSurvey.png}}
%       \caption{Relative fishing mortality (relF) for the DFO Summer RV (RV41) survey and landings. (Left) Results from bayesian change point analysis to determine the timing of %a change in relative F as a result of changing productivity from increased survey biomass. Upper panel represent the posterior means of the bayesian change point model (red %line) along with the input values for relative F. Lower panel represents the probability of a change point occurring at specific time (black line). (Right) Relative F time %series along with the running median (red line) and the medians for the full time series (green) and the lower productivity period (blue). }
%\end{figure}
%
%\begin{figure}
%\centering
%        \subfloat{\includegraphics[width=0.65\textwidth]{\e BCPrelFNEFSCSPRINGSurvey.png}}
%        \subfloat{\includegraphics[width=0.7\textwidth]{\e relFNEFSCSPRING.png}}
%       \caption{Relative F for the NEFSC Spring survey (NSpr41) and landings. (Left) Results from bayesian change point analysis to determine the timing of a change in relative F %as a result of changing productivity from increased survey biomass. Upper panel represent the posterior means of the bayesian change point model (red line) along with the %input values for relative F. Lower panel represents the probability of a change point occurring at specific time (black line). (Right) Relative F time series along with %the running median (red line) and the medians for the full time series (green) and the lower productivity period (blue). }
%\end{figure}
%
%
%\begin{figure}
%\centering
%        \subfloat{\includegraphics[width=0.65\textwidth]{\e BCPrelFNEFSCFALLSurvey.png}}
%        \subfloat{\includegraphics[width=0.7\textwidth]{\e relFNEFSCFALL.png}}
%       \caption{Relative F for the NEFSC Fall survey (NAut41) and landings. (Left) Results from bayesian change point analysis to determine the timing of a change in relative F %as a result of changing productivity from increased survey biomass. Upper panel represent the posterior means of the bayesian change point model (red line) along with the %input values for relative F. Lower panel represents the probability of a change point occurring at specific time (black line). (Right) Relative F time series along with %the running median (red line) and the medians for the full time series (green) and the lower productivity period (blue). Purple line represents the $F_ref$ from the %replacement ratio method. }
%\end{figure}
%
%
%\begin{figure}
%\centering
%        \subfloat{\includegraphics[width=0.65\textwidth]{\e BCPrelFDFOGeorgesSurvey.png}}
%        \subfloat{\includegraphics[width=0.7\textwidth]{\e relFDFOGeorges.png}}
%       \caption{Relative F for the Georges Bank (GB) survey and landings. (Left) Results from bayesian change point analysis to determine the timing of a change in relative F as %a result of changing productivity from increased survey biomass. Upper panel represent the posterior means of the bayesian change point model (red line) along with the %input values for relative F. Lower panel represents the probability of a change point occurring at specific time (black line). (Right) Relative F time series along with %the running median (red line) and the medians for the full time series (green) and the lower productivity period (blue). }
%\end{figure}
%
%
%
%
%
%\end{landscape}
%
%%%%%%%%%%%%%%%%%%%%%%%%%%%AIM plots
%
%\begin{figure}
%\centering
%
%        \includegraphics[width=0.7\textwidth]{\e AIMDFOSummer.png}
%       \caption{Relationship between log(replacement ratio) and log(relative F) from the RV41 survey and landings from 1981 to 2015 (points). Black line represents the robust %regression with an $R^2=0.03$.}
%
%
%\end{figure}
%
%
%\begin{figure}
%\centering
%
%        \includegraphics[width=0.7\textwidth]{\e AIMNEFSCSpring.png}
%       \caption{Relationship between log(replacement ratio) and log(relative F) from the NSpr41 survey and landings from 1981 to 2015 (points). Black line represents the robust %regression with an $R^2=0.015$.}
%
%
%\end{figure}
%
%
%
%\begin{figure}
%\centering
%
%        \includegraphics[width=0.7\textwidth]{\e AIMNEFSCFALL.png} \\
%
%        \includegraphics[width=0.7\textwidth]{\e AIMFREFNEFSCFALLSurvey.png} \\
%
%       \caption{((Upper)Relationship between log(replacement ratio) and log(relative F) from the NAut41 survey and landings from 1981 to 2015 (points). Black line represents the %robust regression with an $R^2=0.11$.(Lower) Bootstrapped histogram of Fref levels from robust regressions.}
%
%
%\end{figure}
%
%
%
%\begin{figure}
%\centering
%
%        \includegraphics[width=0.7\textwidth]{\e AIMDFOGeorges.png}
%       \caption{Relationship between log(replacement ratio) and log(relative F) from the GB survey and landings from 1987 to 2015 (points). Black line represents the robust %regression with an $R^2=0.014$.}
%
%
%\end{figure}
%

%stratified analysis
%numbers
\begin{figure}
\centering
\subfloat{\includegraphics[width=0.5\textwidth]{\e lfa41DFObasenumbers.png}}
\subfloat{\includegraphics[width=0.5\textwidth]{\e lfa41DFOrestratifiednumbers.png}}\\
\subfloat{\includegraphics[width=0.5\textwidth]{\e adjcentlfa41DFOrestratifiednumbers.png}}
\subfloat{\includegraphics[width=0.5\textwidth]{\e lfa41DFOrestratifiednumbersNOY.png}}\\

\caption{DFO Summer RV survey American lobster stratified mean number per tow for the base survey (topleft), restratified survey (topright) restratified to adjacent areas (bottom) from 1970 to 2015. Red line represents a three year running median. Confidence bounds are presented for each point estimate. }
\end{figure}
\clearpage

\begin{figure}
\centering
\subfloat{\includegraphics[width=0.5\textwidth]{\e lfa41NEFSCSpringbasenumbers.png}}
\subfloat{\includegraphics[width=0.5\textwidth]{\e lfa41NEFSCSpringrestratifiednumbers.png}}\\
\subfloat{\includegraphics[width=0.5\textwidth]{\e adjacentlfa41NEFSCSpringrestratifiednumbers.png}}
\subfloat{\includegraphics[width=0.5\textwidth]{\e lfa41NEFSCSpringrestratifiednumbersNOY.png}}\\

\caption{NEFSC spring survey American lobster stratified mean number per tow for the base survey (topleft), restratified survey (topright) restratified to adjacent areas (bottom) from 1969 to 2015. Red line represents a three year running median. Confidence bounds are presented for each point estimate. }
\end{figure}
\clearpage


\begin{figure}
\centering
\subfloat{\includegraphics[width=0.5\textwidth]{\e lfa41NEFSCFallbasenumbers.png}}
\subfloat{\includegraphics[width=0.5\textwidth]{\e lfa41NEFSCFallrestratifiednumbers.png}}\\
\subfloat{\includegraphics[width=0.5\textwidth]{\e adjacentlfa41NEFSCFallrestratifiednumbers.png}}
\subfloat{\includegraphics[width=0.5\textwidth]{\e lfa41NEFSCFallrestratifiednumbersNOY.png}}\\

\caption{NEFSC Fall survey American lobster stratified mean number per tow for the base survey (topleft), restratified survey (topright) restratified to adjacent areas (bottom) from 1969 to 2015. Red line represents a three year running median. Confidence bounds are presented for each point estimate. }
\end{figure}
\clearpage

\begin{figure}

    \includegraphics[width=1\textwidth]{\e lfa41georgesnumbers.png}
    \caption{DFO RV Georges Bank survey American lobster stratified mean number per tow for the Canadian portion of the survey from 1987 to 2015. Red line represents a three year running median. Confidence bounds are presented for each point estimate.}

\end{figure}

\begin{landscape}
\begin{figure}
\centering
    \includegraphics[width=.58\textwidth]{\e pezzackRVDFObasenumbers.png} 
    \includegraphics[width=.58\textwidth]{\e pezzacksummerstratamap.pdf} 
    
    \caption{DFO RV Summer survey American lobster stratified mean number per tow (left) using the strata definitions of Pezzack et al. (2015) (shaded yellow; right) from 1970 to 2015. Red line represents a three year running median. Confidence bounds are presented for each point estimate.}

\end{figure}
\end{landscape}


\begin{landscape}
\begin{figure}
\centering
    \includegraphics[width=.58\textwidth]{\e pezzacklfa41georgesnumbers.png} 
    \includegraphics[width=.58\textwidth]{\e pezzackgeorgesmap41.pdf} 
    
    \caption{DFO Georges survey American lobster stratified mean number per tow (left) using the strata definitions of Pezzack et al. (2015) (shaded yellow; right) from 1987 to 2015. Red line represents a three year running median. Confidence bounds are presented for each point estimate.}

\end{figure}
\end{landscape}



%weights
\begin{figure}
\centering
\subfloat{\includegraphics[width=0.5\textwidth]{\e lfa41DFObaseweights.png}}
\subfloat{\includegraphics[width=0.5\textwidth]{\e lfa41DFOrestratifiedweights.png}}\\
\subfloat{\includegraphics[width=0.5\textwidth]{\e adjacentlfa41DFOrestratifiedweights.png}}
\subfloat{\includegraphics[width=0.5\textwidth]{\e lfa41DFOrestratifiedweightsNOY.png}}\\

\caption{DFO Summer RV survey American lobster stratified mean weight per tow for the base survey (topleft), restratified survey (topright) restratified to adjacent areas bottom) from 1970 to 2015. Red line represents a three year running median. Confidence bounds are presented for each point estimate.}
\end{figure}
\clearpage



\begin{figure}
\centering
\subfloat{\includegraphics[width=0.5\textwidth]{\e lfa41NEFSCSpringbaseweights.png}}
\subfloat{\includegraphics[width=0.5\textwidth]{\e lfa41NEFSCSpringrestratifiedweights.png}}\\
\subfloat{\includegraphics[width=0.5\textwidth]{\e adjacentlfa41NEFSCSpringrestratifiedweights.png}}
\subfloat{\includegraphics[width=0.5\textwidth]{\e lfa41NEFSCSpringrestratifiedweightsNOY.png}}\\

\caption{NEFSC spring survey American lobster stratified mean weight per tow for the base survey (topleft), restratified survey (topright) restratified to adjacent areas (ottom) from 1969 to 2015. Red line represents a three year running median. Confidence bounds are presented for each point estimate. }
\end{figure}
\clearpage



\begin{figure}
\centering
\subfloat{\includegraphics[width=0.5\textwidth]{\e lfa41NEFSCFallbaseweights.png}}
\subfloat{\includegraphics[width=0.5\textwidth]{\e lfa41NEFSCFallrestratifiedweights.png}}\\
\subfloat{\includegraphics[width=0.5\textwidth]{\e adjacentlfa41NEFSCFallrestratifiedweights.png}}
\subfloat{\includegraphics[width=0.5\textwidth]{\e lfa41NEFSCFallrestratifiedweightsNOY.png}}\\

\caption{NEFSC Fall survey American lobster stratified mean weight per tow for the base survey (topleft), restratified survey (topright) restratified to adjacent areas (bottom) from 1969 to 2015. Red line represents a three year running median. Confidence bounds are presented for each point estimate. }
\end{figure}
\clearpage


\begin{figure}

    \includegraphics[width=1\textwidth]{\e lfa41georgesweights.png}
    \caption{DFO RV Georges Bank survey American lobster stratified mean weight per per tow for the Canadian portion of the survey from 1987 to 2015. Red line represents a three year running median. Confidence bounds are presented for each point estimate.}

\end{figure}


%recruits
\begin{figure}
\centering
\subfloat{\includegraphics[width=0.5\textwidth]{\e lfa41DFObasenumbersrecruits.png}}
\subfloat{\includegraphics[width=0.5\textwidth]{\e lfa41DFOrestratifiednumbersrecruits.png}}\\
\subfloat{\includegraphics[width=0.5\textwidth]{\e adjcentlfa41DFOrestratifiednumbersrecruits.png}}

\caption{DFO Summer RV survey American lobster stratified mean number of recruits per tow for the base survey (topleft), restratified survey (topright) restratified to adjacent areas (bottom) from 1999 to 2015. Red line represents a three year running median. Confidence bounds are presented for each point estimate.}
\end{figure}
\clearpage



\begin{figure}
\centering
\subfloat{\includegraphics[width=0.5\textwidth]{\e lfa41NEFSCSpringbasenumbersrecruits.png}}
\subfloat{\includegraphics[width=0.5\textwidth]{\e lfa41NEFSCSpringrestratifiednumbersrecruits.png}}\\
\subfloat{\includegraphics[width=0.5\textwidth]{\e adjacentlfa41NEFSCSpringrestratifiednumbersrecruits.png}}

\caption{NEFSC spring survey American lobster stratified mean number of recruits per tow for the base survey (topleft), restratified survey (topright) restratified to adjacent areas (bottom) from 1969 to 2015. Red line represents a three year running median. Confidence bounds are presented for each point estimate. }
\end{figure}
\clearpage



\begin{figure}
\centering
\subfloat{\includegraphics[width=0.5\textwidth]{\e lfa41NEFSCFallbasenumbersrecruits.png}}
\subfloat{\includegraphics[width=0.5\textwidth]{\e lfa41NEFSCFallrestratifiednumbersrecruits.png}}\\
\subfloat{\includegraphics[width=0.5\textwidth]{\e adjacentlfa41NEFSCFallrestratifiednumbersrecruits.png}}

\caption{NEFSC Fall survey American lobster stratified mean number of recruits per tow for the base survey (topleft), restratified survey (topright) restratified to adjacent areas (bottom) from 1969 to 2015. Red line represents a three year running median. Confidence bounds are presented for each point estimate. }
\end{figure}
\clearpage


\begin{figure}

    \includegraphics[width=1\textwidth]{\e lfa41georgesnumbersrecruits.png}
    \caption{DFO RV Georges Bank survey American lobster stratified number of recruits number per per tow for the Canadian portion of the survey from 2007 to 2015. Red line represents a three year running median. Confidence bounds are presented for each point estimate.}

\end{figure}



%stratified analysis
%commercial 
\begin{figure}
\centering
\subfloat{\includegraphics[width=0.5\textwidth]{\e lfa41DFObasenumberscommercial.png}}
\subfloat{\includegraphics[width=0.5\textwidth]{\e lfa41DFOrestratifiednumberscommercial.png}}\\
\subfloat{\includegraphics[width=0.5\textwidth]{\e adjacentlfa41DFOrestratifiednumberscommercial.png}}
\caption{DFO Summer RV survey commercial sized American lobster stratified mean number per tow for the base survey (topleft), restratified survey (topright) restratified to adjacent areas (bottom) from 1970 to 2015. Red line represents a three year running median. Confidence bounds are presented for each point estimate. Blue line represents the annual number of commercial lobster observed within the survey and area.}
\end{figure}
\clearpage

\begin{figure}
\centering
\subfloat{\includegraphics[width=0.5\textwidth]{\e lfa41NEFSCSpringbasenumberscommercial.png}}
\subfloat{\includegraphics[width=0.5\textwidth]{\e lfa41NEFSCSpringrestratifiednumberscommercial.png}}\\
\subfloat{\includegraphics[width=0.5\textwidth]{\e adjacentlfa41NEFSCSpringrestratifiednumberscommercial.png}}
\caption{NEFSC spring survey commercial sized American lobster stratified mean number per tow for the base survey (topleft), restratified survey (topright) restratified to adjacent areas (bottom) from 1969 to 2015. Red line represents a three year running median. Confidence bounds are presented for each point estimate. Blue line represents the annual number of commercial lobster observed within the survey and area.}
\end{figure}
\clearpage


\begin{figure}
\centering
\subfloat{\includegraphics[width=0.5\textwidth]{\e lfa41NEFSCFallbasenumberscommercial.png}}
\subfloat{\includegraphics[width=0.5\textwidth]{\e lfa41NEFSCFallrestratifiednumberscommercial.png}}\\
\subfloat{\includegraphics[width=0.5\textwidth]{\e adjacentlfa41NEFSCFallrestratifiednumberscommercial.png}}
\caption{NEFSC Fall survey commercial sized American lobster stratified mean number per tow for the base survey (topleft), restratified survey (topright) restratified to adjacent areas (bottom) from 1969 to 2015. Red line represents a three year running median. Confidence bounds are presented for each point estimate. Blue line represents the annual number of commercial lobster observed within the survey and area.}
\end{figure}
\clearpage

\begin{figure}

    \includegraphics[width=1\textwidth]{\e lfa41georgesnumberscommercial.png}
    \caption{DFO RV Georges Bank survey commercial sized American lobster stratified mean number per tow for the Canadian portion of the survey from 1987 to 2015. Red line represents a three year running median. Confidence bounds are presented for each point estimate. Blue line represents the annual number of commercial lobster observed within the survey and area.}

\end{figure}



%stratified analysis
%large female numbers
\begin{figure}
\centering
\subfloat{\includegraphics[width=0.5\textwidth]{\e lfa41DFObasenumberslargefemale.png}}
\subfloat{\includegraphics[width=0.5\textwidth]{\e lfa41DFOrestratifiednumberslargefemale.png}}\\
\subfloat{\includegraphics[width=0.5\textwidth]{\e adjcentlfa41DFOrestratifiednumberslargefemale.png}}
\caption{DFO Summer RV survey large female ($\geq$ 140mm) American lobster stratified mean number per tow for the base survey (topleft), restratified survey (topright) restratified to adjacent areas (bottom) from 1970 to 2015. Red line represents a three year running median. Confidence bounds are presented for each point estimate. Blue points represents the annual number of female lobster $\geq$ 140mm observed within the survey and area.}
\end{figure}
\clearpage

\begin{figure}
\centering
\subfloat{\includegraphics[width=0.5\textwidth]{\e lfa41NEFSCSpringbasenumberslargefemale.png}}
\subfloat{\includegraphics[width=0.5\textwidth]{\e lfa41NEFSCSpringrestratifiednumberslargefemale.png}}\\
\subfloat{\includegraphics[width=0.5\textwidth]{\e adjacentlfa41NEFSCSpringrestratifiednumberslargefemale.png}}
\caption{NEFSC spring survey large female ($\geq$ 140mm) American lobster stratified mean number per tow for the base survey (topleft), restratified survey (topright) restratified to adjacent areas (bottom) from 1969 to 2015. Red line represents a three year running median. Confidence bounds are presented for each point estimate. Blue points represents the annual number of female lobster $\geq$ 140mm observed within the survey and area.}
\end{figure}
\clearpage


\begin{figure}
\centering
\subfloat{\includegraphics[width=0.5\textwidth]{\e lfa41NEFSCFallbasenumberslargefemale.png}}
\subfloat{\includegraphics[width=0.5\textwidth]{\e lfa41NEFSCFallrestratifiednumberslargefemale.png}}\\
\subfloat{\includegraphics[width=0.5\textwidth]{\e adjacentlfa41NEFSCFallrestratifiednumberslargefemale.png}}
\caption{NEFSC Fall survey large female ($\geq$ 140mm) American lobster stratified mean number per tow for the base survey (topleft), restratified survey (topright) restratified to adjacent areas (bottom) from 1969 to 2015. Red line represents a three year running median. Confidence bounds are presented for each point estimate. Blue points represents the annual number of female lobster $\geq$ 140mm observed within the survey and area.}
\end{figure}
\clearpage

\begin{figure}

    \includegraphics[width=1\textwidth]{\e lfa41georgesnumberslargefemale.png}
    \caption{DFO RV Georges Bank survey large female ($\geq$ 140mm) American lobster stratified mean number per tow for the Canadian portion of the survey from 2007 to 2015. Red line represents a three year running median. Confidence bounds are presented for each point estimate. Blue points represents the annual number of female lobster $\geq$ 140mm observed within the survey and area.}

\end{figure}


\begin{landscape}
\begin{figure}
\centering
    \includegraphics[width=.58\textwidth]{\e pezzacklfa41largefemale.png} 
    \includegraphics[width=.58\textwidth]{\e pezzack480481map.pdf} 
    
    \caption{DFO Summer RV survey American lobster stratified mean number per tow for large females ($\geq$ 140) (left) using the strata definitions of Pezzack et al. (2015) (shaded yellow; right) from 1999 to 2015. Red line represents a three year running median. Confidence bounds are presented for each point estimate.}

\end{figure}
\end{landscape}


%dwao
\begin{figure}
\centering
\subfloat{\includegraphics[width=0.5\textwidth]{\e lfa41DFObaseDWAO.png}}
\subfloat{\includegraphics[width=0.5\textwidth]{\e lfa41DFOrestratifiedDWAO.png}}\\
\subfloat{\includegraphics[width=0.5\textwidth]{\e adjacentlfa41DFOrestratifiedDWAO.png}}
\caption{DFO Summer RV survey American lobster design weighted area occupied (DWAO) for the base survey (topleft), restratified survey (topright) restratified to adjacent areas (bottom) from 1970 to 2015. Red line represents a three year running median.}
\end{figure}
\clearpage


\begin{figure}
\centering
\subfloat{\includegraphics[width=0.5\textwidth]{\e lfa41NEFSCSpringbaseDWAO.png}}
\subfloat{\includegraphics[width=0.5\textwidth]{\e lfa41NEFSCSpringrestratifiedDWAO.png}}\\
\subfloat{\includegraphics[width=0.5\textwidth]{\e adjacentlfa41NEFSCSpringrestratifiedDWAO.png}}
\caption{NEFSC spring survey American lobster design weighted area occupied (DWAO) for the base survey (topleft), restratified survey (topright) restratified to adjacent areas (bottom) from 1969 to 2015. Red line represents a three year running median. }
\end{figure}
\clearpage


\begin{figure}
\centering
\subfloat{\includegraphics[width=0.5\textwidth]{\e lfa41NEFSCFallbaseDWAO.png}}
\subfloat{\includegraphics[width=0.5\textwidth]{\e lfa41NEFSCFallrestratifiedDWAO.png}}\\
\subfloat{\includegraphics[width=0.5\textwidth]{\e adjacentlfa41NEFSCFallrestratifiedDWAO.png}}
\caption{NEFSC Fall survey American lobster design weighted area occupied (DWAO) for the base survey (topleft), restratified survey (topright) restratified to adjacent areas (bottom) from 1969 to 2015. Red line represents a three year running median. }
\end{figure}
\clearpage


\begin{figure}

    \includegraphics[width=1\textwidth]{\e lfa41georgesDWAO.png}
    \caption{DFO RV Georges Bank survey American lobster design weighted area occupied for the Canadian portion of the survey from 1987 to 2015. Red line represents a three year running median. }

\end{figure}



%gini
\begin{figure}
\centering
\subfloat{\includegraphics[width=0.5\textwidth]{\e lfa41DFObasegini.png}}
\subfloat{\includegraphics[width=0.5\textwidth]{\e lfa41DFOrestratifiedgini.png}}\\
\subfloat{\includegraphics[width=0.5\textwidth]{\e adjacentlfa41DFOrestratifiedgini.png}}
\caption{DFO Summer RV survey American lobster Gini index of spatial evenness for the base survey (topleft), restratified survey (topright) restratified to adjacent areas (bottom) from 1970 to 2015. Red line represents a three year running median. Breaks in the three year running median are for years where no American lobster were captured in the survey strata.}
\end{figure}
\clearpage


\begin{figure}
\centering
\subfloat{\includegraphics[width=0.5\textwidth]{\e lfa41NEFSCSpringbasegini.png}}
\subfloat{\includegraphics[width=0.5\textwidth]{\e lfa41NEFSCSpringrestratifiedgini.png}}\\
\subfloat{\includegraphics[width=0.5\textwidth]{\e adjacentlfa41NEFSCSpringrestratifiedgini.png}}
\caption{NEFSC spring survey American lobster Gini coefficient of spatial evenness for the base survey (topleft), restratified survey (topright) restratified to adjacent areas (bottom) from 1969 to 2015. Red line represents a three year running median.  }
\end{figure}
\clearpage



\begin{figure}
\centering
\subfloat{\includegraphics[width=0.5\textwidth]{\e lfa41NEFSCFallbasegini.png}}
\subfloat{\includegraphics[width=0.5\textwidth]{\e lfa41NEFSCFallrestratifiedgini.png}}\\
\subfloat{\includegraphics[width=0.5\textwidth]{\e adjacentlfa41NEFSCFallrestratifiedgini.png}}
\caption{NEFSC Fall survey American lobster Gini coefficient of spatial evenness for the base survey (topleft), restratified survey (topright) restratified to adjacent areas (bottom) from 1969 to 2015. Red line represents a three year running median.  }
\end{figure}
\clearpage


\begin{figure}

    \includegraphics[width=1\textwidth]{\e lfa41georgesgini.png}
    \caption{DFO RV Georges Bank survey American lobster Gini coefficient of spatial evenness for the Canadian portion of the survey from 1987 to 2015. Red line represents a three year running median. }

\end{figure}
\end{document}
%sex ratios
%\begin{figure}
%\centering
%\subfloat{\includegraphics[width=0.5\textwidth]{\e sexLFA41baseSummerRV.png}}
%\subfloat{\includegraphics[width=0.5\textwidth]{\e sexLFA41polygonSummerRV.png}}\\
%\subfloat{\includegraphics[width=0.5\textwidth]{\e sexLFA41adjacentpolygonSummerRV.png}}
%\caption{DFO Summer RV survey American lobster proportion females across all size classes for %the base survey (topleft), restratified survey (topright) restratified to adjacent areas (%bottom) from 1970 to 2015. Red line represents a three year running median. Breaks in the three %year running median are for years where no American lobster were captured in the survey strata. %Within each plot the bars represent the sample size of observed lobster within that year and %area.}
%\end{figure}
%\clearpage
%
%
%\begin{figure}
%\centering
%\subfloat{\includegraphics[width=0.5\textwidth]{\e sexLFA41NEFSCspringbase.png}}
%\subfloat{\includegraphics[width=0.5\textwidth]{\e sexLFA41NEFSCspringrestratified.png}}\\
%\subfloat{\includegraphics[width=0.5\textwidth]{\e sexLFA41NEFSCspringadjrestratified.png}}
%\caption{NEFSC spring survey American lobster proportion females across all size classes for %the base survey (topleft), restratified survey (topright) restratified to adjacent areas (%bottom) from 1969 to 2015. Red line represents a three year running median. Within each plot %the bars represent the sample size of observed lobster within that year and area. }
%\end{figure}
%\clearpage
%
%
%
%\begin{figure}
%\centering
%\subfloat{\includegraphics[width=0.5\textwidth]{\e sexLFA41NEFSCfallbase.png}}
%\subfloat{\includegraphics[width=0.5\textwidth]{\e sexLFA41NEFSCfallrestratified.png}}\\
%\subfloat{\includegraphics[width=0.5\textwidth]{\e sexLFA41NEFSCfalladjrestratified.png}}
%\caption{NEFSC Fall survey American lobster proportion females across all size classesfor the %base survey (topleft), restratified survey (topright) restratified to adjacent areas (bottom) %from 1969 to 2015. Red line represents a three year running median. Within each plot the bars %represent the sample size of observed lobster within that year and area. }
%\end{figure}
%\clearpage
%
%
%\begin{figure}
%
%    \includegraphics[width=1\textwidth]{\e sexLFA41dfogeorges.png}
%    \caption{DFO RV Georges Bank survey American lobster proportion females across all size %classes for the Canadian portion of the survey from 1987 to 2015. Red line represents a %three year running median. Bars represent the sample size of observed lobster within that %year and area.}
%
%\end{figure}

%mature sex ratios

%sex ratios
\begin{figure}
\centering
\subfloat{\includegraphics[width=0.5\textwidth]{\e maturesexLFA41baseSummerRV.png}}
\subfloat{\includegraphics[width=0.5\textwidth]{\e maturesexLFA41polygonSummerRV.png}}\\
\subfloat{\includegraphics[width=0.5\textwidth]{\e maturesexLFA41adjacentpolygonSummerRV.png}}
\caption{DFO Summer RV survey American lobster proportion females from mature component ($ \ge 95$ mm) for the base survey (topleft), restratified survey (topright) restratified to adjacent areas (bottom) from 1970 to 2015. Red line represents a three year running median. Breaks in the three year running median are for years where no American lobster were captured in the survey strata. Within each plot the bars represent the sample size of observed lobster within that year and area.}
\end{figure}
\clearpage


\begin{figure}
\centering
\subfloat{\includegraphics[width=0.5\textwidth]{\e maturesexLFA41NEFSCspringbase.png}}
\subfloat{\includegraphics[width=0.5\textwidth]{\e maturesexLFA41NEFSCspringrestratified.png}}\\
\subfloat{\includegraphics[width=0.5\textwidth]{\e maturesexLFA41NEFSCspringadjrestratified.png}}
\caption{NEFSC spring survey American lobster proportion females from mature component ($ \ge 95$ mm) for the base survey (topleft), restratified survey (topright) restratified to adjacent areas (bottom) from 1969 to 2015. Red line represents a three year running median. Within each plot the bars represent the sample size of observed lobster within that year and area. }
\end{figure}
\clearpage



\begin{figure}
\centering
\subfloat{\includegraphics[width=0.5\textwidth]{\e maturesexLFA41NEFSCfallbase.png}}
\subfloat{\includegraphics[width=0.5\textwidth]{\e maturesexLFA41NEFSCfallrestratified.png}}\\
\subfloat{\includegraphics[width=0.5\textwidth]{\e maturesexLFA41NEFSCfalladjrestratified.png}}
\caption{NEFSC Fall survey American lobster proportion females from mature component ($ \ge 95$ mm) for the base survey (topleft), restratified survey (topright) restratified to adjacent areas (bottom) from 1969 to 2015. Red line represents a three year running median. Within each plot the bars represent the sample size of observed lobster within that year and area. }
\end{figure}
\clearpage


\begin{figure}

    \includegraphics[width=1\textwidth]{\e maturesexLFA41dfogeorges.png}
    \caption{DFO RV Georges Bank survey American lobster proportion females from mature component ($ \ge 95$ mm) for the Canadian portion of the survey from 2007 to 2015. Red line represents a three year running median. Bars represent the sample size of observed lobster within that year and area.}

\end{figure}

%immature sex ratios

%sex ratios
\begin{figure}
\centering
\subfloat{\includegraphics[width=0.5\textwidth]{\e immaturesexLFA41baseSummerRV.png}}
\subfloat{\includegraphics[width=0.5\textwidth]{\e immaturesexLFA41polygonSummerRV.png}}\\
\subfloat{\includegraphics[width=0.5\textwidth]{\e immaturesexLFA41adjacentpolygonSummerRV.png}}
\caption{DFO Summer RV survey American lobster proportion females from immature component (\textless 95 mm) for the base survey (topleft), restratified survey (topright) restratified to adjacent areas (bottom) from 1970 to 2015. Red line represents a three year running median. Breaks in the three year running median are for years where no American lobster were captured in the survey strata. Within each plot the bars represent the sample size of observed lobster within that year and area.}
\end{figure}
\clearpage


\begin{figure}
\centering
\subfloat{\includegraphics[width=0.5\textwidth]{\e immaturesexLFA41NEFSCspringbase.png}}
\subfloat{\includegraphics[width=0.5\textwidth]{\e immaturesexLFA41NEFSCspringrestratified.png}}\\
\subfloat{\includegraphics[width=0.5\textwidth]{\e immaturesexLFA41NEFSCspringadjrestratified.png}}
\caption{NEFSC spring survey American lobster proportion females from immature component (\textless 95 mm) for the base survey (topleft), restratified survey (topright) restratified to adjacent areas (bottom) from 1969 to 2015. Red line represents a three year running median. Within each plot the bars represent the sample size of observed lobster within that year and area. }
\end{figure}
\clearpage



\begin{figure}
\centering
\subfloat{\includegraphics[width=0.5\textwidth]{\e immaturesexLFA41NEFSCfallbase.png}}
\subfloat{\includegraphics[width=0.5\textwidth]{\e immaturesexLFA41NEFSCfallrestratified.png}}\\
\subfloat{\includegraphics[width=0.5\textwidth]{\e immaturesexLFA41NEFSCfalladjrestratified.png}}
\caption{NEFSC Fall survey American lobster proportion females from immature component (\textless 95 mm) for the base survey (topleft), restratified survey (topright) restratified to adjacent areas (bottom) from 1969 to 2015. Red line represents a three year running median. Within each plot the bars represent the sample size of observed lobster within that year and area. }
\end{figure}
\clearpage


\begin{figure}

    \includegraphics[width=1\textwidth]{\e immaturesexLFA41dfogeorges.png}
    \caption{DFO RV Georges Bank survey American lobster proportion females from immature component (\textless 95 mm) for the Canadian portion of the survey from 2007 to 2015. Red line represents a three year running median. Bars represent the sample size of observed lobster within that year and area.}

\end{figure}

% Reproductive Potential

NEED TO READD THESE IN FUCK


% Size frequencies

\begin{figure}
\centering

\subfloat{\includegraphics[clip,trim={0 2.1cm 0 1.9cm},width=0.4\textwidth]{\e LengthFrequenciesLFA41baseSummerRV\D 1999\D 2004.pdf}}\\
\subfloat{\includegraphics[clip,trim={0 2.1cm 0 1.9cm},width=0.4\textwidth]{\e LengthFrequenciesLFA41baseSummerRV\D 2005\D 2009.pdf}}\\
\subfloat{\includegraphics[clip,trim={0 2.1cm 0 1.9cm},width=0.4\textwidth]{\e LengthFrequenciesLFA41baseSummerRV\D 2010\D 2014.pdf}}\\
\subfloat{\includegraphics[clip,trim={0 1.5cm 0 1.9cm},width=0.4\textwidth]{\e LengthFrequenciesLFA41baseSummerRV\D 2015\D 2015.pdf}}\\
 \caption{Carapace length frequencies of American lobster captured during the Summer RV survey with the base strata for LFA 41. Bars represents the mean number per tow for each length bin scaled to the maximum numbers per tow. For plots with multiple years bars represent the average over respective time spans. Dashed red line indicates minimum legal size.}
\end{figure}
\clearpage

\begin{figure}
\centering

\subfloat{\includegraphics[clip,trim={0 2.1cm 0 1.9cm},width=0.4\textwidth]{\e LengthFrequenciesLFA41polygonSummerRV\D 1999\D 2004.pdf}}\\
\subfloat{\includegraphics[clip,trim={0 2.1cm 0 1.9cm},width=0.4\textwidth]{\e LengthFrequenciesLFA41polygonSummerRV\D 2005\D 2009.pdf}}\\
\subfloat{\includegraphics[clip,trim={0 2.1cm 0 1.9cm},width=0.4\textwidth]{\e LengthFrequenciesLFA41polygonSummerRV\D 2010\D 2014.pdf}}\\
\subfloat{\includegraphics[clip,trim={0 1.5cm 0 1.9cm},width=0.4\textwidth]{\e LengthFrequenciesLFA41polygonSummerRV\D 2015\D 2015.pdf}}\\
 \caption{Carapace length frequencies of American lobster captured during the Summer RV survey with following the restratification strategy to areas within LFA 41. Bars represents the mean number per tow for each length bin scaled to the maximum numbers per tow. For plots with multiple years bars represent the average over respective time spans. Dashed red line indicates the minimum legal size.}
\end{figure}
\clearpage


\begin{figure}
\centering

\subfloat{\includegraphics[clip,trim={0 2.1cm 0 1.9cm},width=0.4\textwidth]{\e LengthFrequenciesLFA41adjacentpolygonSummerRV\D 1999\D 2004.pdf}}\\
\subfloat{\includegraphics[clip,trim={0 2.1cm 0 1.9cm},width=0.4\textwidth]{\e LengthFrequenciesLFA41adjacentpolygonSummerRV\D 2005\D 2009.pdf}}\\
\subfloat{\includegraphics[clip,trim={0 2.1cm 0 1.9cm},width=0.4\textwidth]{\e LengthFrequenciesLFA41adjacentpolygonSummerRV\D 2010\D 2014.pdf}}\\
\subfloat{\includegraphics[clip,trim={0 1.5cm 0 1.9cm},width=0.4\textwidth]{\e LengthFrequenciesLFA41adjacentpolygonSummerRV\D 2015\D 2015.pdf}}\\
 \caption{Carapace length frequencies of American lobster captured during the Summer RV survey with following the restratification strategy to areas adjacent to LFA 41. Bars represents the mean number per tow for each length bin scaled to the maximum numbers per tow. For plots with multiple years bars represent the average over respective time spans. Dashed red line indicates the minimum legal size.}
\end{figure}
\clearpage

\begin{figure}

\subfloat{\includegraphics[clip,trim={0 2.1cm 0.3cm 2.1cm},width=0.37\textwidth]{\e LengthFrequenciesLFA41NEFSCspringbase\D 1969\D 1974.pdf}}
\subfloat{\includegraphics[clip,trim={0 2.1cm 0.3cm 2.1cm},width=0.37\textwidth]{\e LengthFrequenciesLFA41NEFSCspringbase\D 1990\D 1994.pdf}}
\subfloat{\includegraphics[clip,trim={0 2.1cm 0.3cm 2.1cm},width=0.37\textwidth]{\e LengthFrequenciesLFA41NEFSCspringbase\D 2010\D 2014.pdf}}\\

\subfloat{\includegraphics[clip,trim={0 2.1cm 0.3cm 2.1cm},width=0.37\textwidth]{\e LengthFrequenciesLFA41NEFSCspringbase\D 1975\D 1979.pdf}}
\subfloat{\includegraphics[clip,trim={0 2.1cm 0.3cm 2.1cm},width=0.37\textwidth]{\e LengthFrequenciesLFA41NEFSCspringbase\D 1995\D 1999.pdf}}
\subfloat{\includegraphics[clip,trim={0 2.1cm 0.3cm 2.1cm},width=0.37\textwidth]{\e LengthFrequenciesLFA41NEFSCspringbase\D 2015\D 2015.pdf}}\\

\subfloat{\includegraphics[clip,trim={0 2.1cm 0.3cm 2.1cm},width=0.37\textwidth]{\e LengthFrequenciesLFA41NEFSCspringbase\D 1980\D 1984.pdf}}
\subfloat{\includegraphics[clip,trim={0 2.1cm 0.3cm 2.1cm},width=0.37\textwidth]{\e LengthFrequenciesLFA41NEFSCspringbase\D 2000\D 2004.pdf}}\\
\subfloat{\includegraphics[clip,trim={0 1.4cm 0.3cm 2.1cm},width=0.37\textwidth]{\e LengthFrequenciesLFA41NEFSCspringbase\D 1985\D 1989.pdf}}
\subfloat{\includegraphics[clip,trim={0 1.4cm 0.3cm 2.1cm},width=0.37\textwidth]{\e LengthFrequenciesLFA41NEFSCspringbase\D 2005\D 2009.pdf}}

 \caption{Carapace length frequencies of American lobster captured during the Spring NEFSC survey with the base strata for LFA 41. Bars represents the mean number per tow for each length bin scaled to the maximum numbers per tow. For plots with multiple years bars represent the average over respective time spans. Dashed red line indicates minimum legal size.}
\end{figure}
\clearpage

\begin{figure}
\subfloat{\includegraphics[clip,trim={0 2.1cm 0.3cm 2.1cm},width=0.37\textwidth]{\e LengthFrequenciesLFA41NEFSCspringrestratified\D 1969\D 1974.pdf}}
\subfloat{\includegraphics[clip,trim={0 2.1cm 0.3cm 2.1cm},width=0.37\textwidth]{\e LengthFrequenciesLFA41NEFSCspringrestratified\D 1990\D 1994.pdf}}
\subfloat{\includegraphics[clip,trim={0 2.1cm 0.3cm 2.1cm},width=0.37\textwidth]{\e LengthFrequenciesLFA41NEFSCspringrestratified\D 2010\D 2014.pdf}}\\

\subfloat{\includegraphics[clip,trim={0 2.1cm 0.3cm 2.1cm},width=0.37\textwidth]{\e LengthFrequenciesLFA41NEFSCspringrestratified\D 1975\D 1979.pdf}}
\subfloat{\includegraphics[clip,trim={0 2.1cm 0.3cm 2.1cm},width=0.37\textwidth]{\e LengthFrequenciesLFA41NEFSCspringrestratified\D 1995\D 1999.pdf}}
\subfloat{\includegraphics[clip,trim={0 2.1cm 0.3cm 2.1cm},width=0.37\textwidth]{\e LengthFrequenciesLFA41NEFSCspringrestratified\D 2015\D 2015.pdf}}\\

\subfloat{\includegraphics[clip,trim={0 2.1cm 0.3cm 2.1cm},width=0.37\textwidth]{\e LengthFrequenciesLFA41NEFSCspringrestratified\D 1980\D 1984.pdf}}
\subfloat{\includegraphics[clip,trim={0 2.1cm 0.3cm 2.1cm},width=0.37\textwidth]{\e LengthFrequenciesLFA41NEFSCspringrestratified\D 2000\D 2004.pdf}}\\
\subfloat{\includegraphics[clip,trim={0 1.4cm 0.3cm 2.1cm},width=0.37\textwidth]{\e LengthFrequenciesLFA41NEFSCspringrestratified\D 1985\D 1989.pdf}}
\subfloat{\includegraphics[clip,trim={0 1.4cm 0.3cm 2.1cm},width=0.37\textwidth]{\e LengthFrequenciesLFA41NEFSCspringrestratified\D 2005\D 2009.pdf}}

 \caption{Carapace length frequencies of American lobster captured during the Spring NEFSC survey with the restratified strata for LFA 41. Bars represents the mean number per tow for each length bin scaled to the maximum numbers per tow. For plots with multiple years bars represent the average over respective time spans. Dashed red line indicates minimum legal size.}
\end{figure}
\clearpage


\begin{figure}
\subfloat{\includegraphics[clip,trim={0 2.1cm 0.3cm 2.1cm},width=0.37\textwidth]{\e LengthFrequenciesLFA41NEFSCspringadjrestratified\D 1969\D 1974.pdf}}
\subfloat{\includegraphics[clip,trim={0 2.1cm 0.3cm 2.1cm},width=0.37\textwidth]{\e LengthFrequenciesLFA41NEFSCspringadjrestratified\D 1990\D 1994.pdf}}
\subfloat{\includegraphics[clip,trim={0 2.1cm 0.3cm 2.1cm},width=0.37\textwidth]{\e LengthFrequenciesLFA41NEFSCspringadjrestratified\D 2010\D 2014.pdf}}\\

\subfloat{\includegraphics[clip,trim={0 2.1cm 0.3cm 2.1cm},width=0.37\textwidth]{\e LengthFrequenciesLFA41NEFSCspringadjrestratified\D 1975\D 1979.pdf}}
\subfloat{\includegraphics[clip,trim={0 2.1cm 0.3cm 2.1cm},width=0.37\textwidth]{\e LengthFrequenciesLFA41NEFSCspringadjrestratified\D 1995\D 1999.pdf}}
\subfloat{\includegraphics[clip,trim={0 2.1cm 0.3cm 2.1cm},width=0.37\textwidth]{\e LengthFrequenciesLFA41NEFSCspringadjrestratified\D 2015\D 2015.pdf}}\\

\subfloat{\includegraphics[clip,trim={0 2.1cm 0.3cm 2.1cm},width=0.37\textwidth]{\e LengthFrequenciesLFA41NEFSCspringadjrestratified\D 1980\D 1984.pdf}}
\subfloat{\includegraphics[clip,trim={0 2.1cm 0.3cm 2.1cm},width=0.37\textwidth]{\e LengthFrequenciesLFA41NEFSCspringadjrestratified\D 2000\D 2004.pdf}}\\
\subfloat{\includegraphics[clip,trim={0 1.4cm 0.3cm 2.1cm},width=0.37\textwidth]{\e LengthFrequenciesLFA41NEFSCspringadjrestratified\D 1985\D 1989.pdf}}
\subfloat{\includegraphics[clip,trim={0 1.4cm 0.3cm 2.1cm},width=0.37\textwidth]{\e LengthFrequenciesLFA41NEFSCspringadjrestratified\D 2005\D 2009.pdf}}

 \caption{Carapace length frequencies of American lobster captured during the Spring NEFSC survey with the restratified strata adjacent to LFA 41. Bars represents the mean number per tow for each length bin scaled to the maximum numbers per tow. For plots with multiple years bars represent the average over respective time spans. Dashed red line indicates minimum legal size.}
\end{figure}
\clearpage


\begin{figure}

\subfloat{\includegraphics[clip,trim={0 2.1cm 0.3cm 2.1cm},width=0.37\textwidth]{\e LengthFrequenciesLFA41NEFSCfallbase\D 1969\D 1974.pdf}}
\subfloat{\includegraphics[clip,trim={0 2.1cm 0.3cm 2.1cm},width=0.37\textwidth]{\e LengthFrequenciesLFA41NEFSCfallbase\D 1990\D 1994.pdf}}
\subfloat{\includegraphics[clip,trim={0 2.1cm 0.3cm 2.1cm},width=0.37\textwidth]{\e LengthFrequenciesLFA41NEFSCfallbase\D 2010\D 2014.pdf}}\\

\subfloat{\includegraphics[clip,trim={0 2.1cm 0.3cm 2.1cm},width=0.37\textwidth]{\e LengthFrequenciesLFA41NEFSCfallbase\D 1975\D 1979.pdf}}
\subfloat{\includegraphics[clip,trim={0 2.1cm 0.3cm 2.1cm},width=0.37\textwidth]{\e LengthFrequenciesLFA41NEFSCfallbase\D 1995\D 1999.pdf}}
\subfloat{\includegraphics[clip,trim={0 2.1cm 0.3cm 2.1cm},width=0.37\textwidth]{\e LengthFrequenciesLFA41NEFSCfallbase\D 2015\D 2015.pdf}}\\

\subfloat{\includegraphics[clip,trim={0 2.1cm 0.3cm 2.1cm},width=0.37\textwidth]{\e LengthFrequenciesLFA41NEFSCfallbase\D 1980\D 1984.pdf}}
\subfloat{\includegraphics[clip,trim={0 2.1cm 0.3cm 2.1cm},width=0.37\textwidth]{\e LengthFrequenciesLFA41NEFSCfallbase\D 2000\D 2004.pdf}}\\
\subfloat{\includegraphics[clip,trim={0 1.4cm 0.3cm 2.1cm},width=0.37\textwidth]{\e LengthFrequenciesLFA41NEFSCfallbase\D 1985\D 1989.pdf}}
\subfloat{\includegraphics[clip,trim={0 1.4cm 0.3cm 2.1cm},width=0.37\textwidth]{\e LengthFrequenciesLFA41NEFSCfallbase\D 2005\D 2009.pdf}}

 \caption{Carapace length frequencies of American lobster captured during the fall NEFSC survey with the base strata for LFA 41. Bars represents the mean number per tow for each length bin scaled to the maximum numbers per tow. For plots with multiple years bars represent the average over respective time spans. Dashed red line indicates minimum legal size.}
\end{figure}
\clearpage

\begin{figure}
\subfloat{\includegraphics[clip,trim={0 2.1cm 0.3cm 2.1cm},width=0.37\textwidth]{\e LengthFrequenciesLFA41NEFSCfallrestratified\D 1969\D 1974.pdf}}
\subfloat{\includegraphics[clip,trim={0 2.1cm 0.3cm 2.1cm},width=0.37\textwidth]{\e LengthFrequenciesLFA41NEFSCfallrestratified\D 1990\D 1994.pdf}}
\subfloat{\includegraphics[clip,trim={0 2.1cm 0.3cm 2.1cm},width=0.37\textwidth]{\e LengthFrequenciesLFA41NEFSCfallrestratified\D 2010\D 2014.pdf}}\\

\subfloat{\includegraphics[clip,trim={0 2.1cm 0.3cm 2.1cm},width=0.37\textwidth]{\e LengthFrequenciesLFA41NEFSCfallrestratified\D 1975\D 1979.pdf}}
\subfloat{\includegraphics[clip,trim={0 2.1cm 0.3cm 2.1cm},width=0.37\textwidth]{\e LengthFrequenciesLFA41NEFSCfallrestratified\D 1995\D 1999.pdf}}
\subfloat{\includegraphics[clip,trim={0 2.1cm 0.3cm 2.1cm},width=0.37\textwidth]{\e LengthFrequenciesLFA41NEFSCfallrestratified\D 2015\D 2015.pdf}}\\

\subfloat{\includegraphics[clip,trim={0 2.1cm 0.3cm 2.1cm},width=0.37\textwidth]{\e LengthFrequenciesLFA41NEFSCfallrestratified\D 1980\D 1984.pdf}}
\subfloat{\includegraphics[clip,trim={0 2.1cm 0.3cm 2.1cm},width=0.37\textwidth]{\e LengthFrequenciesLFA41NEFSCfallrestratified\D 2000\D 2004.pdf}}\\
\subfloat{\includegraphics[clip,trim={0 1.4cm 0.3cm 2.1cm},width=0.37\textwidth]{\e LengthFrequenciesLFA41NEFSCfallrestratified\D 1985\D 1989.pdf}}
\subfloat{\includegraphics[clip,trim={0 1.4cm 0.3cm 2.1cm},width=0.37\textwidth]{\e LengthFrequenciesLFA41NEFSCfallrestratified\D 2005\D 2009.pdf}}

 \caption{Carapace length frequencies of American lobster captured during the fall NEFSC survey with the restratified strata for LFA 41. Bars represents the mean number per tow for each length bin scaled to the maximum numbers per tow. For plots with multiple years bars represent the average over respective time spans. Dashed red line indicates minimum legal size.}
\end{figure}
\clearpage


\begin{figure}
\subfloat{\includegraphics[clip,trim={0 2.1cm 0.3cm 2.1cm},width=0.37\textwidth]{\e LengthFrequenciesLFA41NEFSCfalladjrestratified\D 1969\D 1974.pdf}}
\subfloat{\includegraphics[clip,trim={0 2.1cm 0.3cm 2.1cm},width=0.37\textwidth]{\e LengthFrequenciesLFA41NEFSCfalladjrestratified\D 1990\D 1994.pdf}}
\subfloat{\includegraphics[clip,trim={0 2.1cm 0.3cm 2.1cm},width=0.37\textwidth]{\e LengthFrequenciesLFA41NEFSCfalladjrestratified\D 2010\D 2014.pdf}}\\

\subfloat{\includegraphics[clip,trim={0 2.1cm 0.3cm 2.1cm},width=0.37\textwidth]{\e LengthFrequenciesLFA41NEFSCfalladjrestratified\D 1975\D 1979.pdf}}
\subfloat{\includegraphics[clip,trim={0 2.1cm 0.3cm 2.1cm},width=0.37\textwidth]{\e LengthFrequenciesLFA41NEFSCfalladjrestratified\D 1995\D 1999.pdf}}
\subfloat{\includegraphics[clip,trim={0 2.1cm 0.3cm 2.1cm},width=0.37\textwidth]{\e LengthFrequenciesLFA41NEFSCfalladjrestratified\D 2015\D 2015.pdf}}\\

\subfloat{\includegraphics[clip,trim={0 2.1cm 0.3cm 2.1cm},width=0.37\textwidth]{\e LengthFrequenciesLFA41NEFSCfalladjrestratified\D 1980\D 1984.pdf}}
\subfloat{\includegraphics[clip,trim={0 2.1cm 0.3cm 2.1cm},width=0.37\textwidth]{\e LengthFrequenciesLFA41NEFSCfalladjrestratified\D 2000\D 2004.pdf}}\\
\subfloat{\includegraphics[clip,trim={0 1.4cm 0.3cm 2.1cm},width=0.37\textwidth]{\e LengthFrequenciesLFA41NEFSCfalladjrestratified\D 1985\D 1989.pdf}}
\subfloat{\includegraphics[clip,trim={0 1.4cm 0.3cm 2.1cm},width=0.37\textwidth]{\e LengthFrequenciesLFA41NEFSCfalladjrestratified\D 2005\D 2009.pdf}}

 \caption{Carapace length frequencies of American lobster captured during the fall NEFSC survey with the restratified strata adjacent to LFA 41. Bars represents the mean number per tow for each length bin scaled to the maximum numbers per tow. For plots with multiple years bars represent the average over respective time spans. Dashed red line indicates minimum legal size.}
\end{figure}
\clearpage


\begin{figure}
\centering
\subfloat{\includegraphics[clip,trim={0 2cm 0.3cm 2.1cm},width=0.6\textwidth]{\e LengthFrequenciesLFA41dfogeorges\D 2007\D 2014.pdf}}\\
\subfloat{\includegraphics[clip,trim={0 1.4cm 0.3cm 2.1cm},width=0.6\textwidth]{\e LengthFrequenciesLFA41dfogeorges\D 2015\D 2015.pdf}}

 \caption{Carapace length frequencies of American lobster captured during the Georges Bank survey within LFA 41. Bars represents the mean number per tow for each length bin scaled to the maximum numbers per tow. For plots with multiple years bars represent the average over respective time spans. Dashed red line indicates minimum legal size.}
\end{figure}
\clearpage



%Median Length 


\begin{figure}
\centering
\subfloat{\includegraphics[width=0.5\textwidth]{\e medianL\D LengthFrequenciesLFA41baseSummerRV.png}}
\subfloat{\includegraphics[width=0.5\textwidth]{\e medianL\D LengthFrequenciesLFA41polygonSummerRV.png}}\\
\subfloat{\includegraphics[width=0.5\textwidth]{\e medianL\D LengthFrequenciesLFA41adjacentpolygonSummerRV.png}}
\caption{DFO Summer RV survey American lobster population weighted median length (solid line and points) with accompanying first and third quartiles (shaded polygon) for the base survey (topleft), restratified survey (topright) restratified to adjacent areas (bottom) from 1999 to 2015. Red line represents a three year running median. Within each plot the bars represent the sample size of observed lobster within that year and area.}
\end{figure}
\clearpage


\begin{figure}
\centering
\subfloat{\includegraphics[width=0.5\textwidth]{\e medianL\D LengthFrequenciesLFA41NEFSCspringbase.png}}
\subfloat{\includegraphics[width=0.5\textwidth]{\e medianL\D LengthFrequenciesLFA41NEFSCspringrestratified.png}}\\
\subfloat{\includegraphics[width=0.5\textwidth]{\e medianL\D LengthFrequenciesLFA41NEFSCspringadjrestratified.png}}
\caption{NEFSC spring survey American lobster population weighted median length (solid line and points) with accompanying first and third quartiles (shaded polygon) for the base survey (topleft), restratified survey (topright) restratified to adjacent areas (bottom) from 1969 to 2015. Red line represents a three year running median. Within each plot the bars represent the sample size of observed lobster within that year and area. }
\end{figure}
\clearpage



\begin{figure}
\centering
\subfloat{\includegraphics[width=0.5\textwidth]{\e medianL\D LengthFrequenciesLFA41NEFSCfallbase.png}}
\subfloat{\includegraphics[width=0.5\textwidth]{\e medianL\D LengthFrequenciesLFA41NEFSCfallrestratified.png}}\\
\subfloat{\includegraphics[width=0.5\textwidth]{\e medianL\D LengthFrequenciesLFA41NEFSCfalladjrestratified.png}}
\caption{NEFSC Fall survey American lobster population weighted median length (solid line and points) with accompanying first and third quartiles (shaded polygon) for the base survey (topleft), restratified survey (topright) restratified to adjacent areas (bottom) from 1969 to 2015. Red line represents a three year running median. Within each plot the bars represent the sample size of observed lobster within that year and area. }
\end{figure}
\clearpage


\begin{figure}

    \includegraphics[width=1\textwidth]{\e medianL\D LengthFrequenciesLFA41dfogeorges.png}
    \caption{DFO RV Georges Bank survey American lobster population weighted median length (solid line and points) with accompanying first and third quartiles (shaded polygon) for the Canadian portion of the survey from 2007 to 2015. Red line represents a three year running median. Bars represent the sample size of observed lobster within that year and area.}

\end{figure}


%Maximum Length 


\begin{figure}
\centering
\subfloat{\includegraphics[width=0.5\textwidth]{\e max95\D LengthFrequenciesLFA41baseSummerRV.pdf}}
\subfloat{\includegraphics[width=0.5\textwidth]{\e max95\D LengthFrequenciesLFA41polygonSummerRV.pdf}}\\
\subfloat{\includegraphics[width=0.5\textwidth]{\e max95\D LengthFrequenciesLFA41adjacentpolygonSummerRV.pdf}}
\caption{DFO Summer RV survey American lobster population weighted upper 95\% length (solid line and points) for the base survey (topleft), restratified survey (topright) restratified to adjacent areas (bottom) from 1999 to 2015. Red line represents a three year running median. }
\end{figure}
\clearpage


\begin{figure}
\centering
\subfloat{\includegraphics[width=0.5\textwidth]{\e max95\D LengthFrequenciesLFA41NEFSCspringbase.pdf}}
\subfloat{\includegraphics[width=0.5\textwidth]{\e max95\D LengthFrequenciesLFA41NEFSCspringrestratified.pdf}}\\
\subfloat{\includegraphics[width=0.5\textwidth]{\e max95\D LengthFrequenciesLFA41NEFSCspringadjrestratified.pdf}}
\caption{NEFSC spring survey American lobster population weighted upper 95\% length (solid line and points) for the base survey (topleft), restratified survey (topright) restratified to adjacent areas (bottom) from 1999 to 2015. Red line represents a three year running median.}
\end{figure}
\clearpage



\begin{figure}
\centering
\subfloat{\includegraphics[width=0.5\textwidth]{\e max95\D LengthFrequenciesLFA41NEFSCfallbase.pdf}}
\subfloat{\includegraphics[width=0.5\textwidth]{\e max95\D LengthFrequenciesLFA41NEFSCfallrestratified.pdf}}\\
\subfloat{\includegraphics[width=0.5\textwidth]{\e max95\D LengthFrequenciesLFA41NEFSCfalladjrestratified.pdf}}
\caption{NEFSC Fall survey American lobster population weighted upper 95\% length (solid line and points) for the base survey (topleft), restratified survey (topright) restratified to adjacent areas (bottom) from 1999 to 2015. Red line represents a three year running median.}
\end{figure}
\clearpage


\begin{figure}

    \includegraphics[width=1\textwidth]{\e max95\D LengthFrequenciesLFA41dfogeorges.pdf}
    \caption{DFO RV Georges Bank survey American lobster population weighted upper 95\% length (solid line and points) for the base survey (topleft), restratified survey (topright) restratified to adjacent areas (bottom) from 1999 to 2015. Red line represents a three year running median.}

\end{figure}




%Length Diversity 


\begin{figure}
\centering
\subfloat{\includegraphics[width=0.5\textwidth]{\e shannon\D LengthFrequenciesLFA41baseSummerRV.pdf}}
\subfloat{\includegraphics[width=0.5\textwidth]{\e shannon\D LengthFrequenciesLFA41polygonSummerRV.pdf}}\\
\subfloat{\includegraphics[width=0.5\textwidth]{\e shannon\D LengthFrequenciesLFA41adjacentpolygonSummerRV.pdf}}
\caption{DFO Summer RV survey American lobster Shannon equitability of length composition (solid line and points) for the base survey (topleft), restratified survey (topright) restratified to adjacent areas (bottom) from 1999 to 2015. Red line represents a three year running median. }
\end{figure}
\clearpage


\begin{figure}
\centering
\subfloat{\includegraphics[width=0.5\textwidth]{\e shannon\D LengthFrequenciesLFA41NEFSCspringbase.pdf}}
\subfloat{\includegraphics[width=0.5\textwidth]{\e shannon\D LengthFrequenciesLFA41NEFSCspringrestratified.pdf}}\\
\subfloat{\includegraphics[width=0.5\textwidth]{\e shannon\D LengthFrequenciesLFA41NEFSCspringadjrestratified.pdf}}
\caption{NEFSC spring survey American lobster Shannon equitability of length composition  (solid line and points) for the base survey (topleft), restratified survey (topright) restratified to adjacent areas (bottom) from 1999 to 2015. Red line represents a three year running median.}
\end{figure}
\clearpage



\begin{figure}
\centering
\subfloat{\includegraphics[width=0.5\textwidth]{\e shannon\D LengthFrequenciesLFA41NEFSCfallbase.pdf}}
\subfloat{\includegraphics[width=0.5\textwidth]{\e shannon\D LengthFrequenciesLFA41NEFSCfallrestratified.pdf}}\\
\subfloat{\includegraphics[width=0.5\textwidth]{\e shannon\D LengthFrequenciesLFA41NEFSCfalladjrestratified.pdf}}
\caption{NEFSC Fall survey American lobster Shannon equitability of length composition (solid line and points) for the base survey (topleft), restratified survey (topright) restratified to adjacent areas (bottom) from 1999 to 2015. Red line represents a three year running median.}
\end{figure}
\clearpage


\begin{figure}

    \includegraphics[width=1\textwidth]{\e shannon\D LengthFrequenciesLFA41dfogeorges.pdf}
    \caption{DFO RV Georges Bank survey American lobster Shannon equitability of length composition (solid line and points) for the base survey (topleft), restratified survey (topright) restratified to adjacent areas (bottom) from 1999 to 2015. Red line represents a three year running median.}

\end{figure}






% habitat associations


\begin{figure}

    \includegraphics[width=1\textwidth]{\e habitatAssociationsDFOsummerbase.pdf}
    \caption{Time series of habitat associations for American lobster as obtained from the RV summer survey base strata between 1970 and 2015. Circles represent the location of maximum deviation of cumulative distributions from catch weighted effort and effort. Filled circles represent statistically significant habitat associations and open circles represent non significant associations. Red line indicates
the median habitat occupied by lobster. Blue line is the median sampled habitat. Shaded polygon in background is the 95th percentile for range of sampled habitat}

\end{figure}



\begin{figure}

    \includegraphics[width=1\textwidth]{\e habitatAssociationsDFOsummerrestratified.pdf}
    \caption{Time series of habitat associations for American lobster as obtained from the RV summer survey series restratified to LFA41 between 1970 and 2015. Circles represent the location of maximum deviation of cumulative distributions from catch weighted effort and effort. Filled circles represent statistically significant habitat associations and open circles represent non significant associations. Red line indicates
the median habitat occupied by lobster. Blue line is the median sampled habitat. Shaded polygon in background is the 95th percentile for range of sampled habitat}

\end{figure}



\begin{figure}

    \includegraphics[width=1\textwidth]{\e habitatAssociationsDFOsummeradjacentrestratified.pdf}
    \caption{Time series of habitat associations for American lobster as obtained from the RV summer survey series restratified to areas adjacent to LFA41 between 1970 and 2015. Circles represent the location of maximum deviation of cumulative distributions from catch weighted effort and effort. Filled circles represent statistically significant habitat associations and open circles represent non significant associations. Red line indicates
the median habitat occupied by lobster. Blue line is the median sampled habitat. Shaded polygon in background is the 95th percentile for range of sampled habitat}

\end{figure}

% habitat associations


\begin{figure}

    \includegraphics[width=1\textwidth]{\e habitatAssociationsNEFSCspringbase.pdf}
    \caption{Time series of habitat associations for American lobster as obtained from the NEFSC spring survey between 1969 and 2015. Circles represent the location of maximum deviation of cumulative distributions from catch weighted effort and effort. Filled circles represent statistically significant habitat associations and open circles represent non significant associations. Red line indicates
the median habitat occupied by lobster. Blue line is the median sampled habitat. Shaded polygon in background is the 95th percentile for range of sampled habitat}

\end{figure}



\begin{figure}

    \includegraphics[width=1\textwidth]{\e habitatAssociationsNEFSCspringrestratified.pdf}
    \caption{Time series of habitat associations for American lobster as obtained from the NEFSC spring survey restratified to LFA41 between 1969 and 2015. Circles represent the location of maximum deviation of cumulative distributions from catch weighted effort and effort. Filled circles represent statistically significant habitat associations and open circles represent non significant associations. Red line indicates the median habitat occupied by lobster. Blue line is the median sampled habitat. Shaded polygon in background is the 95th percentile for range of sampled habitat}

\end{figure}



\begin{figure}

    \includegraphics[width=1\textwidth]{\e habitatAssociationsNEFSCspringadjacentrestratified.pdf}
    \caption{Time series of habitat associations for American lobster as obtained from the NEFSC spring survey series restratified to areas adjacent to LFA41 between 1969 and 2015. Circles represent the location of maximum deviation of cumulative distributions from catch weighted effort and effort. Filled circles represent statistically significant habitat associations and open circles represent non significant associations. Red line indicates
the median habitat occupied by lobster. Blue line is the median sampled habitat. Shaded polygon in background is the 95th percentile for range of sampled habitat}

\end{figure}


% habitat associations


\begin{figure}

    \includegraphics[width=1\textwidth]{\e habitatAssociationsNEFSCfallbase.pdf}
    \caption{Time series of habitat associations for American lobster as obtained from the NEFSC fall survey with base strata between 1969 and 2015. Circles represent the location of maximum deviation of cumulative distributions from catch weighted effort and effort. Filled circles represent statistically significant habitat associations and open circles represent non significant associations. Red line indicates
the median habitat occupied by lobster. Blue line is the median sampled habitat. Shaded polygon in background is the 95th percentile for range of sampled habitat}

\end{figure}



\begin{figure}

    \includegraphics[width=1\textwidth]{\e habitatAssociationsNEFSCfallrestratified.pdf}
    \caption{Time series of habitat associations for American lobster as obtained from the NEFSC fall survey restratified to LFA41 between 1969 and 2015. Circles represent the location of maximum deviation of cumulative distributions from catch weighted effort and effort. Filled circles represent statistically significant habitat associations and open circles represent non significant associations. Red line indicates
the median habitat occupied by lobster. Blue line is the median sampled habitat. Shaded polygon in background is the 95th percentile for range of sampled habitat}

\end{figure}



\begin{figure}

    \includegraphics[width=1\textwidth]{\e habitatAssociationsNEFSCadjacentfallrestratified.pdf}
    \caption{Time series of habitat associations for American lobster as obtained from the NEFSC fall survey  restratified to areas adjacent to LFA41 between 1969 and 2015. Circles represent the location of maximum deviation of cumulative distributions from catch weighted effort and effort. Filled circles represent statistically significant habitat associations and open circles represent non significant associations. Red line indicates
the median habitat occupied by lobster. Blue line is the median sampled habitat. Shaded polygon in background is the 95th percentile for range of sampled habitat}

\end{figure}


\begin{figure}

    \includegraphics[width=1\textwidth]{\e habitatAssociationsgeorges.pdf}
    \caption{Time series of habitat associations for American lobster as obtained from the Georges Bank survey between 1987 and 2015. Circles represent the location of maximum deviation of cumulative distributions from catch weighted effort and effort. Filled circles represent statistically significant habitat associations and open circles represent non significant associations. Red line indicates the median habitat occupied by lobster. Blue line is the median sampled habitat. Shaded polygon in background is the 95th percentile for range of sampled habitat}

\end{figure}




%%Predator Index

\begin{figure}
\centering
    \includegraphics[width=.6\textwidth]{\e LobPredatorsbiomass.png}\\
    \includegraphics[width=.6\textwidth]{\e LobPredatorsabundance.png}\\
    \caption{Time series of biomass (lower) and abundance (upper) of predators of American lobster captured on the western Scotian Shelf during the summer RV survey.}

\end{figure}




%gini fishery
\begin{figure}

    \includegraphics[width=1\textwidth]{\e giniFootprintCPUE.png}
    \caption{Time series of spatial evenness of fishery catch rates (kg \textbackslash TH estimated through the Gini Index for LFA 41. Red line represents the three year running median. Annual catch rates were estimated by grouping fishing trips into 0.05 deg$^2$.}

\end{figure}



%observer length frequencies

\begin{figure}
\centering
    \includegraphics[width=0.45\textwidth]{\e SW\D Browns\D Summer\D 1977\D 1982.pdf}
    \includegraphics[width=0.45\textwidth]{\e SW\D Browns\D Summer\D 1983\D 1994.pdf}\\
    \includegraphics[width=0.45\textwidth]{\e SW\D Browns\D Summer\D 1998\D 2003.pdf}
    \includegraphics[width=0.45\textwidth]{\e SW\D Browns\D Summer\D 2004\D 2013.pdf}\\
    \includegraphics[width=0.45\textwidth]{\e SW\D Browns\D Summer\D 2014\D 2014.pdf}
    
    \caption{Southwestern Browns Bank summer season carapace length frequency histograms binned 3 mm groups. Red dashed line represents minimum legal size of 82.5 mm. Total sample sizes are shown in the legend.}

\end{figure}

\begin{figure}
\centering
    \includegraphics[width=0.45\textwidth]{\e SW\D Browns\D Autumn\D 1978\D 1993.pdf}
    \includegraphics[width=0.45\textwidth]{\e SW\D Browns\D Autumn\D 1998\D 2002.pdf}\\
    \includegraphics[width=0.45\textwidth]{\e SW\D Browns\D Autumn\D 2003\D 2008.pdf}
    \includegraphics[width=0.45\textwidth]{\e SW\D Browns\D Autumn\D 2009\D 2013.pdf}\\
    \includegraphics[width=0.45\textwidth]{\e SW\D Browns\D Autumn\D 2014\D 2014.pdf}
    
    \caption{Southwestern Browns Bank autumn season carapace length frequency histograms binned 3 mm groups. Red dashed line represents minimum legal size of 82.5 mm. Total sample sizes are shown in the legend.}

\end{figure}



\begin{figure}
\centering
    \includegraphics[width=0.45\textwidth]{\e SE\D Browns\D Spring\D 1981\D 1999.pdf}
    \includegraphics[width=0.45\textwidth]{\e SE\D Browns\D Spring\D 2000\D 2006.pdf}\\
    \includegraphics[width=0.45\textwidth]{\e SE\D Browns\D Spring\D 2007\D 2014.pdf}
    \includegraphics[width=0.45\textwidth]{\e SE\D Browns\D Spring\D 2015\D 2015.pdf}\\
    
    \caption{Southeastern Browns Bank spring season carapace length frequency histograms binned 3 mm groups. Red dashed line represents minimum legal size of 82.5 mm. Total sample sizes are shown in the legend.}

\end{figure}


\begin{figure}
\centering
    \includegraphics[width=0.45\textwidth]{\e Georges\D Basin\D Summer\D 1977\D 2000.pdf}
    \includegraphics[width=0.45\textwidth]{\e Georges\D Basin\D Summer\D 2001\D 2009.pdf}\\
    \includegraphics[width=0.45\textwidth]{\e Georges\D Basin\D Summer\D 2010\D 2014.pdf}
    \includegraphics[width=0.45\textwidth]{\e Georges\D Basin\D Summer\D 2015\D 2015.pdf}
    
    \caption{Georges Basin summer season carapace length frequency histograms binned 3 mm groups. Red dashed line represents minimum legal size of 82.5 mm. Total sample sizes are shown in the legend.}

\end{figure}


\begin{figure}
\centering
    \includegraphics[width=0.45\textwidth]{\e Georges\D Basin\D Spring\D 2001\D 2009.pdf}
    \includegraphics[width=0.45\textwidth]{\e Georges\D Basin\D Spring\D 2010\D 2014.pdf}\\
    \includegraphics[width=0.45\textwidth]{\e Georges\D Basin\D Spring\D 2015\D 2015.pdf}
    
    \caption{Georges Basin spring season carapace length frequency histograms binned 3 mm groups. Red dashed line represents minimum legal size of 82.5 mm. Total sample sizes are shown in the legend.}

\end{figure}


\begin{figure}
\centering
    \includegraphics[width=0.45\textwidth]{\e Georges\D Bank\D Summer\D 1977\D 1989.pdf}
    \includegraphics[width=0.45\textwidth]{\e Georges\D Bank\D Summer\D 1991\D 2001.pdf}\\
    \includegraphics[width=0.45\textwidth]{\e Georges\D Bank\D Summer\D 2002\D 2009.pdf}
    \includegraphics[width=0.45\textwidth]{\e Georges\D Bank\D Summer\D 2010\D 2014.pdf}\\
    \includegraphics[width=0.45\textwidth]{\e Georges\D Bank\D Summer\D 2015\D 2015.pdf}\\
    \caption{Georges Bank summer season carapace length frequency histograms binned 3 mm groups. Red dashed line represents minimum legal size of 82.5 mm. Total sample sizes are shown in the legend.}

\end{figure}

%Median Length Observer data

\begin{figure}

    \includegraphics[width=1\textwidth]{\e medianL\D SW\D Browns\D Summer.png}
    \caption{Southwestern Browns Bank summer season median length (black line) with observed 25th and 75th quantiles (shaded polygon) from American lobster observed during at sampling of fishing activities. Red line represents a three year running median, whereas blue vertical lines represent the annual sample sizes.}

\end{figure}

\begin{figure}

    \includegraphics[width=1\textwidth]{\e medianL\D SW\D Browns\D Autumn.png}
    \caption{Southwestern Browns Bank autumn season median length (black line) with observed 25th and 75th quantiles (shaded polygon) from American lobster observed during at sampling of fishing activities. Red line represents a three year running median, whereas blue vertical lines represent the annual sample sizes.}

\end{figure}

\begin{figure}

    \includegraphics[width=1\textwidth]{\e medianL\D SE\D Browns\D Spring.png}
    \caption{Southeastern Browns Bank spring season median length (black line) with observed 25th and 75th quantiles (shaded polygon) from American lobster observed during at sampling of fishing activities. Red line represents a three year running median, whereas blue vertical lines represent the annual sample sizes.}

\end{figure}

\begin{figure}

    \includegraphics[width=1\textwidth]{\e medianL\D Georges\D Basin\D Summer.png}
    \caption{Georges Basin summer season median length (black line) with observed 25th and 75th quantiles (shaded polygon) from American lobster observed during at sampling of fishing activities. Red line represents a three year running median, whereas blue vertical lines represent the annual sample sizes.}

\end{figure}

\begin{figure}

    \includegraphics[width=1\textwidth]{\e medianL\D Georges\D Basin\D Spring.png}
    \caption{Georges Basin spring season median length (black line) with observed 25th and 75th quantiles (shaded polygon) from American lobster observed during at sampling of fishing activities. Red line represents a three year running median, whereas blue vertical lines represent the annual sample sizes.}

\end{figure}

\begin{figure}

    \includegraphics[width=1\textwidth]{\e medianL\D Georges\D Bank\D Summer.png}
    \caption{Georges Bank summer season median length (black line) with observed 25th and 75th quantiles (shaded polygon) from American lobster observed during at sampling of fishing activities. Red line represents a three year running median, whereas blue vertical lines represent the annual sample sizes.}

\end{figure}


\clearpage



%max95 Length Observer data

\begin{figure}

    \includegraphics[width=1\textwidth]{\e max95\D SW\D Browns\D Summer.pdf}
    \caption{Southwestern Browns Bank summer season maximum 95\% quantile from American lobster observed during at sampling of fishing activities. Red line represents a three year running median.}

\end{figure}

\begin{figure}

    \includegraphics[width=1\textwidth]{\e max95\D SW\D Browns\D Autumn.pdf}
    \caption{Southwestern Browns Bank autumn season maximum 95\% quantile from American lobster observed during at sampling of fishing activities. Red line represents a three year running median.}

\end{figure}

\begin{figure}

    \includegraphics[width=1\textwidth]{\e max95\D SE\D Browns\D Spring.pdf}
    \caption{Southeastern Browns Bank spring season maximum 95\% quantile from American lobster observed during at sampling of fishing activities. Red line represents a three year running median.}

\end{figure}

\begin{figure}

    \includegraphics[width=1\textwidth]{\e max95\D Georges\D Basin\D Summer.pdf}
    \caption{Georges Basin summer season maximum 95\% quantile from American lobster observed during at sampling of fishing activities. Red line represents a three year running median.}

\end{figure}

\begin{figure}

    \includegraphics[width=1\textwidth]{\e max95\D Georges\D Basin\D Spring.pdf}
    \caption{Georges Basin spring season maximum 95\% quantile from American lobster observed during at sampling of fishing activities. Red line represents a three year running median.}

\end{figure}

\begin{figure}

    \includegraphics[width=1\textwidth]{\e max95\D Georges\D Bank\D Summer.pdf}
    \caption{Georges Bank summer season maximum 95\% quantile from American lobster observed during at sampling of fishing activities. Red line represents a three year running median.}

\end{figure}


\clearpage


%diversity Length Observer data

\begin{figure}

    \includegraphics[width=1\textwidth]{\e shannon\D SW\D Browns\D Summer.pdf}
    \caption{Southwestern Browns Bank summer season Shannon equitability of length composition from American lobster observed during at sampling of fishing activities. Red line represents a three year running median.}

\end{figure}

\begin{figure}

    \includegraphics[width=1\textwidth]{\e shannon\D SW\D Browns\D Autumn.pdf}
    \caption{Southwestern Browns Bank autumn season Shannon equitability of length composition from American lobster observed during at sampling of fishing activities. Red line represents a three year running median.}

\end{figure}

\begin{figure}

    \includegraphics[width=1\textwidth]{\e shannon\D SE\D Browns\D Spring.pdf}
    \caption{Southeastern Browns Bank spring season Shannon equitability of length composition from American lobster observed during at sampling of fishing activities. Red line represents a three year running median.}

\end{figure}

\begin{figure}

    \includegraphics[width=1\textwidth]{\e shannon\D Georges\D Basin\D Summer.pdf}
    \caption{Georges Basin summer season Shannon equitability of length composition from American lobster observed during at sampling of fishing activities. Red line represents a three year running median.}

\end{figure}

\begin{figure}

    \includegraphics[width=1\textwidth]{\e shannon\D Georges\D Basin\D Spring.pdf}
    \caption{Georges Basin spring season Shannon equitability of length composition from American lobster observed during at sampling of fishing activities. Red line represents a three year running median.}

\end{figure}

\begin{figure}

    \includegraphics[width=1\textwidth]{\e shannon\D Georges\D Bank\D Summer.pdf}
    \caption{Georges Bank summer season Shannon equitability of length composition from American lobster observed during at sampling of fishing activities. Red line represents a three year running median.}

\end{figure}


\clearpage



% environmental conditions


\begin{figure}

    \includegraphics[width=.5\textwidth]{\e lfa41DFOTemp.png}
    \includegraphics[width=.5\textwidth]{\e lfa41georgestemperature.png}\\
    \includegraphics[width=.5\textwidth]{\e lfa41NEFSCFallTemps.png}
    \includegraphics[width=.5\textwidth]{\e lfa41NEFSCSpringTemps.png}\\
    \caption{Stratified mean temperatures from DFO RV summer (upper-left), Georges Bank (upper-right), NEFSC fall (lower-left) and NEFSC spring (lower-right) surveys with base strata for LFA 41. Within each plot red line represents running median and error bars are the 95 \% bootstrapped confidence intervals.}

\end{figure}



\begin{figure}

    \includegraphics[width=1\textwidth]{\e AMO.png}
    \caption{Annual mean anomalies of the Atlantic multidecadal osscillation (AMO). Data obtained from \url{http://www.esrl.noaa.gov/psd/data/correlation//amon.us.long.data}}

\end{figure}


\end{document}

