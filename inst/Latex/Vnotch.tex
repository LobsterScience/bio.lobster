\documentclass{beamer}
 
\mode<presentation> {

%\usetheme{Hannover}
%\usetheme{Boadilla}

\usetheme{AnnArbor}

\usecolortheme{whale}
%\usecolortheme{seagull}

%\usefonttheme{structuresmallcapsserif}

}

\usepackage[utf8]{inputenc}
 \usepackage{graphicx}
\usepackage[toc,page]{appendix}
\usepackage{pdfpages}
\usepackage{array}
\usepackage{animate}
\usepackage{amsmath}
\usepackage{multirow}
\usepackage{adjustbox}
\newcolumntype{C}[1]{>{\centering\let\newline\\\arraybackslash\hspace{0pt}}m{#1}}

\newcommand{\yr}{2016} % current assessment year

%\graphicspath{ {R:/Science/Population Ecology Division/Sections/Eastern Scotian Shelf/Clam Group/Assessment/2016/LatexPresentation/} }
 
%Information to be included in the title page:
\title[Surfclam Assessment]{Assessment of the \\*Arctic Surfclam (\textit{Mactromeris polynyma}) \\*on Banquereau and Grand Bank}
\author{Brad Hubley, Susan Heaslip and Ryan Stanley}
\institute[DFO]{Science Branch, Fisheries and Oceans Canada}
\date{April 20-21, 2017}

 
 
\begin{document}
 
\frame{\titlepage}
 

\begin{frame}
\frametitle{Outline}
\tableofcontents
\end{frame}


%\begin{itemize}
%\item Background
%\item Ecosystem Considerations 
%\item Data Overview
%\item Fishery Data
%\item Survey Data
%\item Habitat suitability
%\item Spatial Management Areas
%\item Spatial Production Model
%\item Discussion: risk assessment, uncertainties, research recomendations
%\end{itemize}

\section{Background}
\subsection{History of the Fishery}
\begin{frame}
\frametitle{Background}
History of the Arctic Surfclam Fishery
\begin{figure}
        \begin{center}
            \includegraphics[width=\textwidth,height=0.7\textheight,keepaspectratio]{timeline2016.png}
   \vspace{1cm}
	\end{center}
   
 \end{figure}

\end{frame}


\subsection{Arctic Surfclam Life History}

\begin{frame}
\frametitle{Background}
Arctic Surfclam Distribution
\begin{figure}
        \begin{center}
            \includegraphics[width=\textwidth,height=0.58\textheight,keepaspectratio]{fao_distn.png}
       
 \end{center}
    \end{figure}


\begin{itemize}
\item Arctic Seas, from Rhode Island (western Atlantic) to Alaska and Puget Sound (east Pacific). Map source: FAO
\end{itemize}

\end{frame}




\begin{frame}
\frametitle{Background}
Arctic Surfclam Life History
\begin{figure}
        \begin{center}
            \includegraphics[width=\textwidth,height=0.4\textheight,keepaspectratio]{surfclam.png}
        \end{center}
    \end{figure}


\begin{itemize}
\item Large (up to 160 mm)
\item Long-lived - can reach more than 60 years of age
\item Reach reproductive maturity between 5-8 years of age
\item Broadcast spawners
\item Benthic substrates with medium to large grain sediments, up to ~100 m depth, and water temperatures less than 15 degrees C
\end{itemize}

\end{frame}


\subsection{Ecosystem Considerations}

\begin{frame}
\frametitle{Background}
Ecosystem Considerations: Impacts of Dredging
\begin{itemize}
\item Large immediate impact on the substrate and benthic organisms
\item Sediment is liquefied down to a minimum of 20 cm 
\item Largest species impact is the removal of clams and other non-target bivalves from harvesting and incidental mortality
\item Harvest efficiencies greater than 90 percent are not uncommon and more than 2/3 of the remaining clams can be damaged
\item Population structure may not return to an unfished state without area closures
\item Low recruitment over the 10 year post-dredging period
\item Areas with high clam biomass are fished more frequently and intensly than other areas, preventing the opportunity for recovery
\item Areas impacted by dredging experiance a shift in species composition such that they are dominanted by colonizing species
\item changes in sediment structure can persist for more than 10 years
\end{itemize}

\end{frame}



\begin{frame}
\frametitle{Background}
Ecosystem Considerations: Bycatch
\begin{figure}
        \begin{center}
            \includegraphics[width=\textwidth,height=0.35\textheight,keepaspectratio]{landed_bycatch.png}
        \end{center}
    \end{figure}


\begin{itemize}
\item Dominated by bivalves: e.g., Greenland cockles, propeller clams, and ocean quahogs
\item Most abundant non-bivalves: whelk and sea cucumbers
\item Discards: Sea Cucumbers (49.3), Arctic Surfclam (37.4), Ocean Quahog (5), Whelks (3.3), Northern Propellerclam (3.3), Greenland Cockle (1.6), and Mussels (0.1)
\end{itemize}

\end{frame}



\section{Secondary Indicators Report}


\begin{frame}
\frametitle{Secondary Indicators Report}
The annual monitoring program for the status of the fishery for Arctic Surfclam in Atlantic Canada is described in the document “Offshore Surfclam Science Monitoring Program”. Three indicators are used to monitor the fishery: 
\begin{itemize}
\item catch per unit effort (CPUE)
\item the spatial extent or footprint of the fishery
\item the abundance of older/larger clams in the catch.
\end{itemize}

\end{frame}

\begin{frame}
\frametitle{Secondary Indicators Report}
\begin{figure}
        \begin{center}
            \includegraphics[width=\textwidth,height=0.6\textheight,keepaspectratio]{BQ_CPUE_Anon.png}
        \end{center}
    \end{figure}

CPUE is declining but still above the threshold level of 70 g/\(m^2\) on Banquereau Bank.
\end{frame}

\begin{frame}
\frametitle{Secondary Indicators Report}
\begin{figure}
        \begin{center}
            \includegraphics[width=\textwidth,height=0.6\textheight,keepaspectratio]{GB_CPUE_Anon.png}
        \end{center}
    \end{figure}

CPUE on Grand Bank has been low in the last decade since the majority of fishing activity has focussed on Banquereau. Recent expansion of activity on the bank have been associated with CPUE values well above the threshold level of 50 g/\(m^2\).
\end{frame}


\begin{frame}
\frametitle{Secondary Indicators Report}
\begin{figure}
        \begin{center}
            \includegraphics[width=\textwidth,height=0.6\textheight,keepaspectratio]{BQ_Footprint.png}
        \end{center}
    \end{figure}

High CPUE values on Banquereau have been supported by landings near the TAC set at 24,000 t in recent years with an associated year over year footprint below the threshold level of 250 \(km^2\).
\end{frame}


\begin{frame}
\frametitle{Secondary Indicators Report}
\begin{figure}
        \begin{center}
            \includegraphics[width=\textwidth,height=0.6\textheight,keepaspectratio]{GB_Footprint.png}
        \end{center}
    \end{figure}

Grand Bank catches have increased from 730 t in 2015 to 13560 t in 2016, but remain below the TAC set at 14756 t. Increased activity on Grand bank has been associated with an increase in the spatial footprint of the fishery, increasing from 7.5 \(km^2\) in 2015 to  95 \(km^2\) in 2016. This footprint remains below the threshold level of 125 \(km^2\).
\end{frame}



\begin{frame}
\frametitle{Secondary Indicators Report}
\begin{figure}
        \begin{center}
            \includegraphics[width=\textwidth,height=0.6\textheight,keepaspectratio]{BQ_PrecLargeClams.png}
        \end{center}
    \end{figure}

The proportion of older/larger Arctic Surfclam in the unsorted catch from the fishery on Banquereau Bank in 2016, as indicated by onboard sampling data provided by industry, was 2.31"%" ≥ 120 mm (Figure 3, Table 2). This value is above the trigger level of 1.0"%" ≥ 120 mm and has increased since 2015 (1.55"%"). 
\end{frame}



\begin{frame}
\frametitle{Secondary Indicators Report}
\begin{figure}
        \begin{center}
            \includegraphics[width=\textwidth,height=0.6\textheight,keepaspectratio]{GB_PrecLargeClams.png}
        \end{center}
    \end{figure}

The proportion of older/larger Arctic Surfclam in the unsorted catch from the fishery on Grand Bank in 2016, as indicated by onboard sampling data provided by industry, was 15.48"%" ≥ 105 mm (Figure 6, Table 6). This value is well above the trigger level of 1.0"%" ≥ 105 mm but has decreased since 2015 (19.17"%").  
\end{frame}



\section{Data Inputs}
\subsection{Fishery Data}

\begin{frame}
\frametitle{Data Inputs}
Fishery Data
\begin{itemize}
\item Catch, Effort and Location 
\item Recorded in log records by watch (every six hours)
\item Vessel Monitoring System (VMS) Data
\begin{itemize}
\item tracking system that transmits GPS location estimates of vessels via satellite on an hourly basis
\item Licence conditions for the Arctic Surfclam fishery require licence holders to carry a DFO Approved VMS unit on their vessel
\item VMS locations were associated with log records to provide more accurate positional information 
\end{itemize}
\item Fishery data is only source of recent and time series data
\end{itemize}

\end{frame}


\begin{frame}
\frametitle{Fishery Data}
Catch and Effort for Banquereau and Grand Bank

\begin{figure}
        \begin{center}
            \includegraphics[width=\textwidth,height=0.8\textheight,keepaspectratio]{CatchEffortPlot.pdf}
        \end{center}
    \end{figure}


\end{frame}

%
\begin{frame}
\frametitle{Fishery Data}
Some issues with CPUE data

\begin{itemize}
    \item Varies spatially, i.e. decreases as an area is locally depleted than increases when effort targets a new area
    \item Biased as industry makes improvements in efficiency
    \item lag in reporting catch, i.e. some of the catch reported in a watch may be the result of effort reported in the previous watch
\end{itemize}


\end{frame}

%
\begin{frame}
\frametitle{Fishery Data}
Censoring catch and effort data to mitigate lag issue

\begin{figure}
        \begin{center}
            \includegraphics[width=\textwidth,height=0.8\textheight,keepaspectratio]{CatchEffortDist.pdf}
        \end{center}
    \end{figure}


\end{frame}



\begin{frame}
\frametitle{Fishery Data}
Distribution of Catch on Banquereau 2004-\yr

\begin{figure}
        \begin{center}
            \includegraphics[width=\textwidth,height=0.8\textheight,keepaspectratio]{totalVMSBanCatch.pdf}
        \end{center}
    \end{figure}


\end{frame}

\begin{frame}
\frametitle{Fishery Data}
Distribution of Effort on Banquereau 2004-\yr

\begin{figure}
        \begin{center}
            \includegraphics[width=\textwidth,height=0.8\textheight,keepaspectratio]{totalVMSBanEffort.pdf}
        \end{center}
    \end{figure}


\end{frame}

\begin{frame}
\frametitle{Fishery Data}
Distribution of CPUE on Banquereau 2004-\yr

\begin{figure}
        \begin{center}
            \includegraphics[width=\textwidth,height=0.8\textheight,keepaspectratio]{totalVMSBanCPUE.pdf}
        \end{center}
    \end{figure}


\end{frame}

\begin{frame}
\frametitle{Fishery Data}
Distribution of Catch on Grand Bank 2004-\yr

\begin{figure}
        \begin{center}
            \includegraphics[width=\textwidth,height=0.8\textheight,keepaspectratio]{GrandTotalVMSGrandCatch.pdf}
        \end{center}
    \end{figure}


\end{frame}
\begin{frame}
\frametitle{Fishery Data}
Length Frequency Distribution of Catch on Banqureau

\begin{figure}
        \begin{center}
            \includegraphics[width=\textwidth,height=0.8\textheight,keepaspectratio]{FisheryLengthFrequencyBanqCatch.pdf}
        \end{center}
    \end{figure}


\end{frame}

\subsection{Survey Data}
\begin{frame}
\frametitle{Survey Data}
IDW interpolation of Surfclam survey density estimates 2004

\begin{figure}
        \begin{center}
            \includegraphics[page=1,width=\textwidth,height=0.8\textheight,keepaspectratio]{SurveyDensity.pdf}
        \end{center}
    \end{figure}


\end{frame}


\begin{frame}
\frametitle{Survey Data}
IDW interpolation of Surfclam Survey density estimates 2010

\begin{figure}
        \begin{center}
            \includegraphics[page=2,width=\textwidth,height=0.8\textheight,keepaspectratio]{SurveyDensity.pdf}
        \end{center}
    \end{figure}


\end{frame}


\begin{frame}
\frametitle{Survey Data}
Comaprison of Survey and Fishery CPUE density estimates

\begin{figure}
        \begin{center}
            \includegraphics[page=1,width=\textwidth,height=0.8\textheight,keepaspectratio]{SurveyCPUEcompare.pdf}
        \end{center}
    \end{figure}


\end{frame}


\begin{frame}
\frametitle{Survey Data}
Comaprison of Survey and Fishery CPUE density estimates

\begin{figure}
        \begin{center}
            \includegraphics[page=2,width=\textwidth,height=0.8\textheight,keepaspectratio]{SurveyCPUEcompare.pdf}
        \end{center}
    \end{figure}


\end{frame}
\section{Habitat Suitability}
\begin{frame}
\frametitle{Habitat Suitability}
VMS Density: Kernal Density Smoother

\begin{figure}
        \begin{center}
            \includegraphics[width=\textwidth,height=0.8\textheight,keepaspectratio]{VMSdensity.pdf}
        \end{center}
    \end{figure}


\end{frame}

\begin{frame}{VMS Animation}

\begin{itemize}
    \item Illustrates movement of fishing vessels from 2004 to \yr
    \item Fishing activity moves around to different areas
    \item Vessels return to the same areas over time
    \item Fishing occurs almost entirely in the "fished areas" 
\end{itemize}

\end{frame}

\frametitle{Spatial Management Areas}
\begin{frame}
\frametitle{Spatial Management Areas}
The Surfclam steering committee requested a small number of areas be delineated to provide an example for how the department could move towards spatial management. Science was asked to consider the following criteria:

\begin{itemize}
    \item easily navigable (made of straight lines)
    \item encompass large scale contiguous clam beds
    \item roughly equal in total biomass
    \item include both high and low density areas
\end{itemize}


\end{frame}


\begin{frame}
\frametitle{Spatial Management Areas}
An illustrative example

\begin{figure}
        \begin{center}
            \includegraphics[width=\textwidth,height=0.8\textheight,keepaspectratio]{NewAreas2.pdf}
        \end{center}
    \end{figure}


\end{frame}


\begin{frame}{}
\frametitle{Spatial Management Areas}
Table 4. Area summary
\centering
\adjustbox{max height=\dimexpr\textheight-5.5cm\relax,
max width=\textwidth}{
\renewcommand{\arraystretch}{2}%
\begin{tabular}{|C{2cm}|C{2cm}|C{2cm}|C{2cm}|C{2cm}|C{2cm}|C{2cm}|C{2cm}|C{2cm}|}
\hline
\multirow{2}{2cm}{\centering Area ID}
& \multirow{2}{2cm}{\parbox{2cm}{\centering Total area (km2)}}
& \multirow{2}{2cm}{\parbox{2cm}{\centering Fished area (km2)}}
& \multirow{2}{2cm}{\parbox{2cm}{\centering Avg. annual catch (t)}}
& \multirow{2}{2cm}{\parbox{2cm}{\centering Total catch since 2004 (t)}}
& \multicolumn{4}{c|}{\centering Biomass estimates} \\
\cline{6-9}
& & & & & 2010 Survey total area
& 2010 Survey fished area
& 2010 CPUE fished area
& \yr CPUE fished area \\ \hline
1 & 3008 & 315 & 3510 & 45628 & 192448 & 24934 & 56127 & 32888 \\
2 & 2008 & 436 & 4587 & 67879 & 182519 & 43281 & 55914 & 48582 \\
3 & 3251 & 442 & 4051 & 53054 & 338452 & 75693 & 49354 & 52571 \\
4 & 1555 & 220 & 2546 & 31281 & 31892 & 10509 & 24703 & 20092 \\
5 & 2078 & 189 & 4907 & 64280 & 138773 & 56719 & 32163 & 25500 \\
Total & 11900 & 1601 & 20163 & 262122 & 884085 & 211136 & 218262 & 179634 \\
\hline
\end{tabular}
}
 
\end{frame}


\begin{frame}{Spatial Management Areas}
Table 6. Catches by area
\centering
\begin{tabular}{|c|c|c|c|c|c|c|}
\hline
& Area 1 & Area 2 & Area 3 & Area 4 & Area 5 & Total \\
\hline
2010 & 2664 & 9574 & 2682 &	393 & 7245 & 22558 \\
\hline
2011 & 4399 & 3953 & 3326 &	5101 & 4079 & 20858 \\
\hline
2012 & 2973 & 4416 & 1333 & 6771 & 4720 & 20214 \\
\hline
2013 & 6213 & 1766. & 865 & 5535 & 4891 & 19270 \\
\hline
2014 & 7744 & 11395 & 230 &	787 & 3483 & 23640 \\
\hline
2015 & 2896 & 7230 & 1766 &	519	& 10876 & 23287 \\
\hline
2016 & 3804 & 12880 & 4461 & 724 & 2284 & 24154 \\
 
\hline
 
\end{tabular}
\end{frame}

\begin{frame}
\frametitle{Spatial Management Areas}
%Commercial Catch Rates (CPUE) by area

\begin{figure}
        \begin{center}
            \includegraphics[width=\textwidth,height=0.85\textheight,keepaspectratio]{CPUE.pdf}
        \end{center}
    \end{figure}


\end{frame}



\begin{frame}
\frametitle{Spatial Management Areas}
CPUE (density) can be expanded to total fished area (grey) to get biomass estimate

\begin{figure}
        \begin{center}
            \includegraphics[width=\textwidth,height=0.8\textheight,keepaspectratio]{NewAreas2.pdf}
        \end{center}
    \end{figure}


\end{frame}


\begin{frame}{Spatial Management Areas}
Table 5. CPUE expanded to fished area to get Biomass (t)
\centering
\begin{tabular}{|c|c|c|c|c|c|c|}
\hline
& Area 1 & Area 2 & Area 3 & Area 4 & Area 5 & Total \\
\hline
2010 & 56127 & 55914 & 49354 & 24703 & 32162 & 218262 \\
\hline
2011 & 448450 &	59094 &	66947 &	39419 &	361901 & 246500 \\
\hline
2012 & 36332 & 57088 & 60280 & 31472 & 37949 & 223121 \\
\hline
2013 & 44056 & 42641 & 62062 & 26989 & 26193 & 201940 \\
\hline
2014 & 34571 & 55034 & 47538 & 25443 & 23673 & 186259 \\
\hline
2015 & 24447 & 41505 & 31827 & 18611 & 22420 & 138811 \\
\hline
2016 & 32888 & 48582 & 52571 & 20092 & 25500 & 179634 \\
\hline
 
\end{tabular}
\end{frame}
 

%\begin{frame}
%\frametitle{Dredge efficiency}
%Previous Assessment corrected for efficiency (0.45) but did not include uncertainty
%
%\begin{figure}
%        \begin{center}
%            \includegraphics[width=\textwidth,height=0.8\textheight,keepaspectratio]{dredgeefficiency.png}
%        \end{center}
%    \end{figure}
%
%\end{frame}

\section{Spatial Production Model}

\begin{frame}
\frametitle{Spatial Production Model}
The Spatial Production Model (SPM) is a surplus production model fit similtaneously to each area $j$ with some parameters shared across areas implemented in a Bayesian state space framework

\begin{itemize}
    \item Requires only Catch and CPUE data
    \item Shares parameters across areas
    \item Incorporates prior information (e.g. dredge efficiency estimates)
    \item Estimates process and observation errors independently
    \item Propogates errors through credible intervals
\end{itemize}


\end{frame}



\begin{frame}
\frametitle{Spatial Production Model}
State equation (subject to process error)

\begin{equation*}
    B_{t+1,j} = B_{t,j} + r_jB_{t,j} \biggl(\frac{B_{t,j}}{K_j} \biggr)
\end{equation*}

Observation equation (subject to observation error)


\begin{equation*}
    O_{t,j} = B_{t,j} * q
\end{equation*}


\end{frame}



\begin{frame}
\frametitle{Spatial Production Model}
$K$, carrying capacity

\begin{itemize}
    \item estimated across areas
    \item scaled by habitat (fished) area
\end{itemize}

\begin{equation*}
    K_j=\bar{K}*\frac{H_j}{\bar{H}}
 \end{equation*}

\end{frame}



\begin{frame}
\frametitle{Spatial Production Model}
Priors for $r$, the intrinsic rate of population growth

\begin{itemize}
    \item estimated within a hierarchical structure by estimated average across areas
    \item Overall average used to construct priors for each area
\end{itemize}

%\begin{equation*}
\begin{gather*}
    \bar{r} \sim {unif(0,1)}\\
    \sigma^2_r \sim {LN(-0.35,0.08)}\\
    r_j \sim {LN(log(\bar{r}),\sigma^2_r)}
 \end{gather*}
   
%\end{equation*}

\end{frame}



\begin{frame}
\frametitle{Spatial Production Model}
Priors for error and dredge efficiency $q$

\begin{itemize}
    \item observation error based on CVs from the CPUE index

    \begin{equation*}
        \frac{1}{\sigma^2_\epsilon} \sim{gamma(shape=3, rate=0.4)}
    \end{equation*}

    \item uniform prior for process

    \begin{equation*}
        \sigma^2_\tau \sim{unif(0,5)}
    \end{equation*}

    \item prior for $q$ based on dredge efficiency experiment
   
    \begin{equation*}
        q \sim{beta(a=6, b=7.33)}
    \end{equation*}

\end{itemize}

\end{frame}


\begin{frame}
\frametitle{Spatial Production Model}
%Fit to CPUE data

\begin{figure}
        \begin{center}
            \includegraphics[width=\textwidth,height=0.8\textheight,keepaspectratio]{SPM1fit.pdf}
        \end{center}
    \end{figure}


\end{frame}



\begin{frame}
\frametitle{Spatial Production Model}
Posteriors for shared parameters

\begin{figure}
        \begin{center}
            \includegraphics[width=\textwidth,height=0.8\textheight,keepaspectratio]{SPM1post_single.pdf}
        \end{center}
    \end{figure}


\end{frame}



\begin{frame}
\frametitle{Spatial Production Model}
%Exploitation estiamtes

\begin{figure}
        \begin{center}
            \includegraphics[width=\textwidth,height=0.8\textheight,keepaspectratio]{SPM1exploitation.pdf}
        \end{center}
    \end{figure}


\end{frame}

\begin{frame}
\frametitle{Spatial Production Model}
%Biomass estiamtes 

\begin{figure}
        \begin{center}
            \includegraphics[width=\textwidth,height=0.8\textheight,keepaspectratio]{SPM1biomass.pdf}
        \end{center}
    \end{figure}


\end{frame}

\begin{frame}{Spatial Production Model}
Table 7. Median biomass estiamtes 
\centering
\begin{tabular}{|c|c|c|c|c|c|c|}
\hline
& Area 1 & Area 2 & Area 3 & Area 4 & Area 5 & Total \\
\hline
2010 & 123966 & 164725 & 147533 & 72839 & 94772 & 603835 \\
\hline
2011 & 116653 & 164424 & 162810 & 83547 & 95132 & 622566 \\
\hline
2012 & 105688 & 160248 & 160312 & 76792 & 93167 & 596207 \\
\hline
2013 & 105230 & 145994 & 154263 & 67725 & 80173 & 553385 \\
\hline
2014 & 91811 & 152498 & 134693 & 60144 & 72519 & 511665 \\
\hline
2015 & 78023 & 137511 & 117572 & 54002 & 70552 & 457660 \\
\hline
2016 & 84091 & 139978 & 132869 & 54377 & 64632 & 475947 \\
\hline
\end{tabular}
\end{frame}
 
\subsection{Reference Points}
\begin{frame}
\frametitle{Reference Points}
%$B_MSY$

\begin{figure}
        \begin{center}
            \includegraphics[page=1,width=\textwidth,height=0.8\textheight,keepaspectratio]{SurfClamRefs.pdf}
        \end{center}
    \end{figure}


\end{frame}



\begin{frame}
\frametitle{Reference Points}
%MSY

\begin{figure}
        \begin{center}
            \includegraphics[page=2,width=\textwidth,height=0.8\textheight,keepaspectratio]{SurfClamRefs.pdf}
        \end{center}
    \end{figure}


\end{frame}



\begin{frame}
\frametitle{Reference Points}
%$F_MSY$

\begin{figure}
        \begin{center}
            \includegraphics[page=3,width=\textwidth,height=0.8\textheight,keepaspectratio]{SurfClamRefs.pdf}
        \end{center}
    \end{figure}


\end{frame}




\begin{frame}
\frametitle{Reference Points}
%Phase Plots

\begin{figure}
        \begin{center}
            \includegraphics[width=\textwidth,height=0.8\textheight,keepaspectratio]{PhasePlots.png}
        \end{center}
    \end{figure}


\end{frame}
\begin{frame}
\frametitle{Spatial Production Model}
%Biomass estiamtes 

\begin{figure}
        \begin{center}
            \includegraphics[width=\textwidth,height=0.8\textheight,keepaspectratio]{SPM1biomass.png}
        \end{center}
    \end{figure}


\end{frame}

\section{Discussion}
 
\begin{frame}{Discussion}

\begin{itemize}
    \item Only fishery CPUE provides a time series of abundance
    \item Previous assessment assumed low F will have negligible effect on biomass 
    \item This analysis indicates fishing has reduced biomass within fished areas
    \item Fishable biomass is defined as the biomass within fished areas 
\end{itemize}

\end{frame}
 

 
\begin{frame}{Conclusions and Advice}
\centering
\adjustbox{max height=\dimexpr\textheight-5.5cm\relax,
max width=\textwidth}{
\renewcommand{\arraystretch}{2}%
\begin{tabular}{|l|l|l|}
\hline
F Level & Fished Area & Total Area \\
\hline
High (~0.09) & High & Extreme \\
\hline
Medium (~0.045) & Medium & Very High \\
\hline
Low (~0.026) & Low & High \\
\hline
\end{tabular}
}

\begin{itemize}
    \item Less information about unfished areas makes it more risky to use total bank area
    \item Additional effort likely to be directed to already fished areas increasing the risk of local over exploitation
    \item Declines in CPUE were observed under F levels that were significantly lower $F_MSY$
    \item The medium risk F level would result in TACs that are comparable to the current TAC 
\end{itemize}


\end{frame}
 
\begin{frame}{Conclusions and Advice}
        \begin{center}
            \includegraphics[width=\textwidth,height=0.8\textheight,keepaspectratio]{TACs.png}
        \end{center}

\end{frame}
 

\begin{frame}{Additional uncertainties}

\begin{itemize}
    \item Bias in CPUE, industry becoming more efficient, $q$ is not stable but likely increases over time
    \item $r$ and $K$ are partly confounded in the SPM, if $r$ is too high and $K$ is too low reference points will be too optimistic (higher $F_MSY$ and lower $B_MSY$)
    \item Subjectivity in defining the fished areas (how fished is fished?) i.e. threshold = 30 pings / \(km^2\)
\end{itemize}

\end{frame}
 
%\begin{frame}{(Some) Research Recomendations}
%
%\begin{itemize}
%    \item Habitat suitability based on environmental variables to substitute for VMS proxy
%    \item Better estimates of commercial dredge efficiency used to improve prior on $q$
%    \item Better spatial resolution through use of VMS allows for depletion effects to be analyzed at local levels 
%    \item Study connectivity between beds (recruitment)
%    \item Standardize and correct for biases in CPUE data
%\end{itemize}
%
%\end{frame}


\end{document}


