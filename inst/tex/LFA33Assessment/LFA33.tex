\documentclass{beamer}
 
\mode<presentation> {

%\usetheme{Hannover}
%\usetheme{Boadilla}

\usetheme{AnnArbor}

\usecolortheme{whale}
%\usecolortheme{seagull}

%\usefonttheme{structuresmallcapsserif}

}

\usepackage[utf8]{inputenc}
 \usepackage{graphicx}
\usepackage{pdfpages}
\usepackage{array}
%\usepackage{animate}
\usepackage{amsmath}
\usepackage{multirow}
\usepackage{adjustbox}
\newcolumntype{C}[1]{>{\centering\let\newline\\\arraybackslash\hspace{0pt}}m{#1}}


\newcommand{\ebh}{\string~/bio.data/bio.lobster/figures/Assessment/LFA33/} %change this to set figure directory
\newcommand{\D}{.}

 
%Information to be included in the title page:
\title[LFA 33]{LFA 33 Assessment 2018}
\author[Hubley, Cook, Howse and Denton]{Brad Hubley, Adam Cook, Victoria Howse, Cheryl Denton}
\institute[DFO]{Science Branch, Fisheries and Oceans Canada}
\date{Oct. 1, 2018}

 
 
\begin{document}
 
\frame{\titlepage}
 


\begin{frame}
\frametitle{Lobster Fishing Area 33}
\begin{figure}
        \begin{center}
            \includegraphics[clip,trim={0 0 0 0cm},width=\textwidth,height=0.8\textheight,keepaspectratio]{\ebh LFA33map.png}
        \end{center}
    \end{figure}
\end{frame}


\begin{frame}
\frametitle{Lobster Fishing Area 33}
%\bftext{Indicators}
\begin{itemize}
\item LFA 27-33 Framework Jan. 2018 
\begin{itemize}
\item Assessment methods were developed
\item Primary, secondary and contextual indicators
\item Reference points on primary indicators
\end{itemize}
\item LFA 33 Assessment (Today)
\begin{itemize}
\item Applies method to most up to date data
\item Scheduled to present the most recent data before the the season opens
\item LFA 27-32 Winter 2019 
\end{itemize}
\end{itemize}

\end{frame}


\begin{frame}
\frametitle{Data}
%\bftext{Indicators}
\begin{itemize}
\item Logbooks 
\begin{itemize}
\item Landings
\item Effort
\end{itemize}
\item Fishermen Science Research Society (FSRS) Recruitment trap data
\begin{itemize}
\item Number, size, sex of captured lobster
\item Temperature loggers
\end{itemize}
\item Sampling, at sea and in port
\begin{itemize}
\item Number, size, sex of captured/landed lobster
\item No recent data
\end{itemize}
\end{itemize}

\end{frame}



\begin{frame}
\frametitle{Assessment}
\begin{itemize}
\item Primary Indicators 
\begin{itemize}
\item Catch Per Unit Effort (CPUE)
\item Exploitation
\end{itemize}
\item Secondary Indicators 
\begin{itemize}
\item Landings and effort
\item FSRS recruitment traps legal and recruit size indices
\end{itemize}
\item Contextual Indicators
\begin{itemize}
\item Size structure
\item Reproductive potential
\item Temperature
\end{itemize}
\end{itemize}

\end{frame}



\begin{frame}
\frametitle{Primary Indicators}
Catch Per Unit Effort (CPUE)
\begin{itemize}
\item not modelled 
\item landings standardized by the number of trap hauls 
\item An indicator of Stock Status 
\item Assumed to be propotional to the exploited lobster population 
\item Reference points 
\end{itemize}
\end{frame}

\begin{frame}
\frametitle{Primary Indicators}
Reference Points
\begin{itemize}
\item CPUE time series (1990-2016) was used to define the upper stock reference (USR) and limit reference point (LRP) 
\item Represents both low and high productivity time periods and covers approximately 2 generations 
\item The median of this time series was used as the proxy for BMSY 
\item USR 80\% of BMSY = 0.28 kg/TH
\item LRP 40\% of BMSY = 0.14 kg/TH
\end{itemize}
\end{frame}


\begin{frame}
\frametitle{Primary Indicators}
CPUE
\begin{figure}
        \begin{center}
            \includegraphics[clip,trim={0 0 0 0cm},width=\textwidth,height=0.8\textheight,keepaspectratio]{\ebh CatchRateRefs33.png}
        \end{center}
    \end{figure}
\end{frame}



\begin{frame}
\frametitle{Primary Indicators}
Exploitation
\begin{itemize}
\item Continuous Change in Ratio (CCIR) method
\item The change in ratio between the number of legal size and just under legal size lobseters captured in FSRS traps as legal sized lobsters are removed throughout the season can be used to estimate exploitation
\end{itemize}
\begin{figure}
        \begin{center}
            \includegraphics[clip,trim={0 0 0 0cm},width=0.4\textwidth,height=0.6\textheight,keepaspectratio]{\ebh predictedLFA33example2016.png}
            \includegraphics[clip,trim={0 0 0 0cm},width=0.4\textwidth,height=0.6\textheight,keepaspectratio]{\ebh exploitationLFA33example2016.png}
        \end{center}
    \end{figure}
\end{frame}


\begin{frame}
\frametitle{Primary Indicators}
Removal Reference
\begin{itemize}
    \setlength\itemsep{2em}
\item The removal reference (RR75) was defined as the 75th quantile of the posterior distribution of the maximum modeled CCIR exploitation rate
\item Given that the regional lobster stocks are currently in a highly productive state and population growth has not decreased under the range of estimated exploitation, it is reasonable to assume RR75 is less than FMSY 
\end{itemize}
\end{frame}


\begin{frame}
\frametitle{Primary Indicators}
CCIR Exploitation 
\begin{figure}
        \begin{center}
            \includegraphics[clip,trim={0 0 0 0cm},width=\textwidth,height=0.8\textheight,keepaspectratio]{\ebh ExploitationRefs33.png}
        \end{center}
    \end{figure}
\end{frame}



\begin{frame}
\frametitle{Secondary Indicators}
\begin{itemize}
    \setlength\itemsep{2em}
\item Presented as time series without reference ponits
\item Landings and effort from logbooks
\item Abundance of Legal and Recruit size lobsters from FSRS recruitment traps 
\end{itemize}
\end{frame}



\begin{frame}
\frametitle{Secondary Indicators}
Landings and Effort
\begin{figure}
        \begin{center}
            \includegraphics[clip,trim={0 0 0 0cm},width=\textwidth,height=0.8\textheight,keepaspectratio]{\ebh FisheryPlot33.png}
        \end{center}
    \end{figure}
\end{frame}



\begin{frame}
\frametitle{Secondary Indicators}
FSRS recruitment traps
\begin{itemize}
\item Time series of Legal (82.5 mm) and Recruit (70 - 82.5 mm) size classes 
\item Bayesian Model to account for variables
\begin{itemize}
\item Legals: temperature and day of season (depletion) 
\item Recruits: temperature and number of Legals (competition) 
\item 95\% Credible intervals from posterior
\end{itemize}
\end{itemize}
\end{frame}

\begin{frame}
\frametitle{Secondary Indicators}
FSRS recruitment traps
\begin{figure}
        \begin{center}
            \includegraphics[clip,trim={0 0 0 0cm},width=\textwidth,height=0.8\textheight,keepaspectratio]{\ebh FSRSRecruitCatchRate33.png}
        \end{center}
    \end{figure}
\end{frame}


\begin{frame}
\frametitle{Contextual Indicators}
\begin{itemize}
\item Describe biological processes that influence production, ecosystem and fishery preformance
\begin{itemize}
\item Abundance 
\item Size Structure (median size, max size, proportion of new recruits)
\item Reproductive Potential (proportion of berried females, proportion of mature)
\item Temperature 
\end{itemize}
\item Multivariate analysis to show how patternes change over time 
\end{itemize}
\end{frame}


\begin{frame}
\frametitle{Contextual Indicators}
Principle Components
\begin{figure}
        \begin{center}
            \includegraphics[clip,trim={0 0 1 0cm},width=\textwidth,height=0.4\textheight,keepaspectratio]{\ebh Indicators\D OrdinationLFA33\D PC1.png}\\
            \includegraphics[clip,trim={0 0 1 0cm},width=\textwidth,height=0.4\textheight,keepaspectratio]{\ebh Indicators\D OrdinationLFA33\D PC2.png}
        \end{center}
    \end{figure}
\end{frame}



\begin{frame}
\frametitle{Contextual Indicators}
Anomalies
\begin{figure}
        \begin{center}
            \includegraphics[clip,trim={0 0 0 0cm},width=\textwidth,height=0.8\textheight,keepaspectratio]{\ebh Indicators\D OrdinationLFA33\D anomalies.png}
        \end{center}
    \end{figure}
\end{frame}



\begin{frame}
\frametitle{Fishery Footprint}
\begin{itemize}
    \setlength\itemsep{2em}
\item Spatial distribution of Landings
\item Grids reported in logbooks
\item Shows increasing landings from offshore grids
\end{itemize}
\end{frame}


\begin{frame}
\frametitle{Fishery Footprint}
\begin{figure}
        \begin{center}
            \includegraphics[clip,trim={0 0 2 0cm},width=\textwidth,height=0.9\textheight,keepaspectratio]{\ebh FisheryFootprint2011.png}
        \end{center}
    \end{figure}
\end{frame}




\begin{frame}
\frametitle{Fishery Footprint}
\begin{figure}
        \begin{center}
            \includegraphics[clip,trim={0 0 2 0cm},width=\textwidth,height=0.9\textheight,keepaspectratio]{\ebh FisheryFootprint2012.png}
        \end{center}
    \end{figure}
\end{frame}




\begin{frame}
\frametitle{Fishery Footprint}
\begin{figure}
        \begin{center}
            \includegraphics[clip,trim={0 0 2 0cm},width=\textwidth,height=0.9\textheight,keepaspectratio]{\ebh FisheryFootprint2013.png}
        \end{center}
    \end{figure}
\end{frame}




\begin{frame}
\frametitle{Fishery Footprint}
\begin{figure}
        \begin{center}
            \includegraphics[clip,trim={0 0 2 0cm},width=\textwidth,height=0.9\textheight,keepaspectratio]{\ebh FisheryFootprint2014.png}
        \end{center}
    \end{figure}
\end{frame}




\begin{frame}
\frametitle{Fishery Footprint}
\begin{figure}
        \begin{center}
            \includegraphics[clip,trim={0 0 2 0cm},width=\textwidth,height=0.9\textheight,keepaspectratio]{\ebh FisheryFootprint2015.png}
        \end{center}
    \end{figure}
\end{frame}




\begin{frame}
\frametitle{Fishery Footprint}
\begin{figure}
        \begin{center}
            \includegraphics[clip,trim={0 0 2 0cm},width=\textwidth,height=0.9\textheight,keepaspectratio]{\ebh FisheryFootprint2016.png}
        \end{center}
    \end{figure}
\end{frame}




\begin{frame}
\frametitle{Fishery Footprint}
\begin{figure}
        \begin{center}
            \includegraphics[clip,trim={0 0 2 0cm},width=\textwidth,height=0.9\textheight,keepaspectratio]{\ebh FisheryFootprint2017.png}
        \end{center}
    \end{figure}
\end{frame}




\begin{frame}
\frametitle{Fishery Footprint}
\begin{figure}
        \begin{center}
            \includegraphics[clip,trim={0 0 2 0cm},width=\textwidth,height=0.9\textheight,keepaspectratio]{\ebh FisheryFootprint2018.png}
        \end{center}
    \end{figure}
\end{frame}



\begin{frame}
\frametitle{Conclusions and Advice}
\begin{itemize}
\item The stock status indicator, CPUE, has increased substantically in the last years 
\item The 3 year running median value for CPUE for the 2017/2018 season is 1.08 kg/TH, which is > 3x the USR (0.28 kg/TH)
\item The 3 year running median value of CCIR exploitation for the 2017/2018 season is 0.56, which is below the RR75 (0.81).
\item Reduction of exploitation in the inshore areas where this data is available. 
\item Fishing effort has moved to more offshore to areas that were not previously heavily exploited and are not monitored for exploitation.
\end{itemize}
\end{frame}



\begin{frame}
\frametitle{Conclusions and Advice}
Precautionary Approach
\begin{figure}
        \begin{center}
            \includegraphics[clip,trim={0 0 0 0cm},width=\textwidth,height=0.8\textheight,keepaspectratio]{\ebh PhasePlot.png}
        \end{center}
    \end{figure}
\end{frame}








\end{document}


