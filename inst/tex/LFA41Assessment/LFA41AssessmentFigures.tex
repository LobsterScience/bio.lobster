\documentclass[11pt]{article}
\usepackage{graphicx}
\usepackage{subfig}
\usepackage{pdfcomment}
\usepackage{amsmath}
\usepackage{lscape}
\usepackage{hyperref}
\usepackage[top=2.4cm, bottom=2.4cm, left=3cm, right=3cm]{geometry}
\usepackage{fancyhdr}
\pagestyle{fancy}

\lhead{\bf Maritimes Region}
\rhead{\bf LFA41 Assessment- 2017}
\lfoot{September 13, 2017}
\cfoot{\thepage}
\renewcommand{\headrulewidth}{0.4pt}
\renewcommand{\footrulewidth}{0.4pt}
\newcommand{\D}{.}
\newcommand{\tl}{\textless}
\newcommand{\e}{/SpinDr/backup/bio_data/bio.lobster/figures/} %change this to set figure directory
\newcommand{\cp}{\caption}

\begin{document}

\begin{landscape}
% maps section
\begin{figure}
\centering
    \pdftooltip{\includegraphics[width=1\textwidth]{\e LFAMapATL.jpg}}{Figure 1}
    \caption{Map of the Lobster Fishing Areas in Atlantic Canada using the boundaries identified in the Atlantic fishery regulations.}

\end{figure}
\end{landscape}


\begin{figure}
\centering
    \pdftooltip{\includegraphics[width=.8\textwidth]{\e newoffshoreareas.jpg}}{Figure 2}
    \caption{Map showing the offshore zones used in assessments. Zone 1 represents Crowell Basin, Zone 2 SW Browns, Zone 3 Georges Basin, Zone 4 SE Browns and Zone 5 Georges Bank. 
}

\end{figure}

\begin{landscape}

\begin{figure}
\centering
\pdftooltip{\subfloat{\includegraphics[width=0.65\textwidth]{\e summerstratamap.pdf}}
\subfloat{\includegraphics[width=0.5\textwidth]{\e summerstrata41closeup.pdf}}}{Figure 3}\\
\caption{ Map of Lobster Fishing Areas (LFAs) in black overlain with the full DFO Summer RV survey strata shown in red (left). Close-up of the fished areas of Lobster Fishing Area 41 (blue line) with the DFO Summer RV survey strata included in survey trends outlined in red (right).}
\end{figure}

\end{landscape}

%survey length frequencies 
\begin{figure}
\centering
    \pdftooltip{\includegraphics[width=0.8\textwidth]{\e DFORVSurveyLengthFreqAllv41.png}}{Figure 4}
    \caption{Comparison of sampled length frequencies from the DFO summer RV survey for the entire surveyed area (red) and the lobsters sampled within LFA14 (black). Densities were scaled to the maximum density within each data set.}
\end{figure}

%survey bubbles
        \begin{figure}
        \centering
    \pdftooltip{\subfloat{\includegraphics[clip,trim={0 2.1cm 0.3cm 2.1cm},width=0.37\textwidth]{\e surveyBubblesDFOSummer\D 1970\D 1975.pdf}}
                \subfloat{\includegraphics[clip,trim={0 2.1cm 0.3cm 2.1cm},width=0.37\textwidth]{\e surveyBubblesDFOSummer\D 1976\D 1980.pdf}}
                \subfloat{\includegraphics[clip,trim={0 2.1cm 0.3cm 2.1cm},width=0.37\textwidth]{\e surveyBubblesDFOSummer\D 1981\D 1985.pdf}}}{Figure 5}\\
                \subfloat{\includegraphics[clip,trim={0 2.1cm 0.3cm 2.1cm},width=0.37\textwidth]{\e surveyBubblesDFOSummer\D 1986\D 1990.pdf}}
                \subfloat{\includegraphics[clip,trim={0 2.1cm 0.3cm 2.1cm},width=0.37\textwidth]{\e surveyBubblesDFOSummer\D 1991\D 1995.pdf}}
                \subfloat{\includegraphics[clip,trim={0 2.1cm 0.3cm 2.1cm},width=0.37\textwidth]{\e surveyBubblesDFOSummer\D 1996\D 2000.pdf}}\\
                \subfloat{\includegraphics[clip,trim={0 2.1cm 0.3cm 2.1cm},width=0.37\textwidth]{\e surveyBubblesDFOSummer\D 2001\D 2005.pdf}}
                \subfloat{\includegraphics[clip,trim={0 2.1cm 0.3cm 2.1cm},width=0.37\textwidth]{\e surveyBubblesDFOSummer\D 2006\D 2010.pdf}}
                \subfloat{\includegraphics[clip,trim={0 2.1cm 0.3cm 2.1cm},width=0.37\textwidth]{\e surveyBubblesDFOSummer\D 2011\D 2015.pdf}}\\
                \subfloat{\includegraphics[clip,trim={0 2.1cm 0.3cm 2.1cm},width=0.37\textwidth]{\e surveyBubblesDFOSummer\D 2016\D 2016.pdf}}\\
        
        
         \caption{Map of the abundance of lobster captured during DFO's summer RV survey of the Scotian Shelf. Strata boundaries are outlined in red and LFA 41 stock boundaries are outlined in blue. Size of the symbols are scaled to the number of lobster observed within each tow. Black points represent tow locations with no lobsters.}
        \end{figure}
        \clearpage



\begin{figure}

    \pdftooltip{\includegraphics[width=1\textwidth]{\e georgesmap41.pdf}}{Figure 6}
    \caption{DFO Georges Bank Spring strata from the depth stratified survey are shown in red and green. The strata outlined in green are those used in survey trends from the Georges Bank Survey. Lobster Fishing Area 41 (blue line) is outlined in blue}

\end{figure}

\begin{landscape}


        \begin{figure}
        \centering
        \pdftooltip{\subfloat{\includegraphics[clip,trim={0 2.1cm 0.3cm 2.1cm},width=0.47\textwidth]{\e surveyBubblesDFOGeorges\D 1987\D 1995.pdf}}}{Figure 7}
                    \subfloat{\includegraphics[clip,trim={0 2.1cm 0.3cm 2.1cm},width=0.47\textwidth]{\e surveyBubblesDFOGeorges\D 1996\D 2000.pdf}}
                    \subfloat{\includegraphics[clip,trim={0 2.1cm 0.3cm 2.1cm},width=0.47\textwidth]{\e surveyBubblesDFOGeorges\D 2001\D 2005.pdf}}\\
                    \subfloat{\includegraphics[clip,trim={0 2.1cm 0.3cm 2.1cm},width=0.47\textwidth]{\e surveyBubblesDFOGeorges\D 2006\D 2010.pdf}}
                    \subfloat{\includegraphics[clip,trim={0 2.1cm 0.3cm 2.1cm},width=0.47\textwidth]{\e surveyBubblesDFOGeorges\D 2011\D 2015.pdf}}
                    \subfloat{\includegraphics[clip,trim={0 2.1cm 0.3cm 2.1cm},width=0.47\textwidth]{\e surveyBubblesDFOGeorges\D 2016\D 2016.pdf}}\\

         \caption{Map of the abundance of lobster captured during DFO's Georges Bank Survey. Strata boundaries are outlined in red and LFA 41 stock boundaries are outlined in blue. Size of the symbols are scaled to the number of lobster observed within each tow. Black points represent tow locations with no lobsters.}
        \end{figure}
        \clearpage

\begin{figure}
\centering
        \pdftooltip{\subfloat{\includegraphics[width=.8\textwidth]{\e americanmap41full.pdf}}
\subfloat{\includegraphics[width=0.8\textwidth]{\e americanmap41.pdf}}}{Figure 8}\\
\caption{NEFSC spring and autumn strata from the depth stratified survey shown in red (left). Lobster Fishing Area 41 (blue line) with the NEFSC spring and autumn strata (shown in red) used for the analysis of survey trends (right).}
\end{figure}


\end{landscape}


        \begin{figure}
        \centering
                \pdftooltip{\subfloat{\includegraphics[clip,trim={0 2.1cm 0.3cm 2.1cm},width=0.37\textwidth]{\e surveyBubblesNEFSCSpring\D 1969\D 1975.pdf}}
                            \subfloat{\includegraphics[clip,trim={0 2.1cm 0.3cm 2.1cm},width=0.37\textwidth]{\e surveyBubblesNEFSCSpring\D 1976\D 1980.pdf}}
                            \subfloat{\includegraphics[clip,trim={0 2.1cm 0.3cm 2.1cm},width=0.37\textwidth]{\e surveyBubblesNEFSCSpring\D 1981\D 1985.pdf}}}{Figure 9}\\
                            \subfloat{\includegraphics[clip,trim={0 2.1cm 0.3cm 2.1cm},width=0.37\textwidth]{\e surveyBubblesNEFSCSpring\D 1986\D 1990.pdf}}
                            \subfloat{\includegraphics[clip,trim={0 2.1cm 0.3cm 2.1cm},width=0.37\textwidth]{\e surveyBubblesNEFSCSpring\D 1991\D 1995.pdf}}
                            \subfloat{\includegraphics[clip,trim={0 2.1cm 0.3cm 2.1cm},width=0.37\textwidth]{\e surveyBubblesNEFSCSpring\D 1996\D 2000.pdf}}\\
                            \subfloat{\includegraphics[clip,trim={0 2.1cm 0.3cm 2.1cm},width=0.37\textwidth]{\e surveyBubblesNEFSCSpring\D 2001\D 2005.pdf}}
                            \subfloat{\includegraphics[clip,trim={0 2.1cm 0.3cm 2.1cm},width=0.37\textwidth]{\e surveyBubblesNEFSCSpring\D 2006\D 2010.pdf}}
                            \subfloat{\includegraphics[clip,trim={0 2.1cm 0.3cm 2.1cm},width=0.37\textwidth]{\e surveyBubblesNEFSCSpring\D 2011\D 2015.pdf}}\\
                            \subfloat{\includegraphics[clip,trim={0 2.1cm 0.3cm 2.1cm},width=0.37\textwidth]{\e surveyBubblesNEFSCSpring\D 2016\D 2016.pdf}}\\
                    

         \caption{Map of the abundance of lobster captured during during NEFSC's Spring Survey of the Gulf of Maine, Georges Bank and Scotian Shelf. Strata boundaries are outlined in red and LFA 41 stock boundaries are outlined in blue. Size of the symbols are scaled to the number of lobster observed within each tow. Black points represent tow locations with no lobsters.} 
        \end{figure}
        \clearpage


        \begin{figure}
        \centering
                \pdftooltip{\subfloat{\includegraphics[clip,trim={0 2.1cm 0.3cm 2.1cm},width=0.37\textwidth]{\e surveyBubblesNEFSCFall\D 1969\D 1975.pdf}}
                            \subfloat{\includegraphics[clip,trim={0 2.1cm 0.3cm 2.1cm},width=0.37\textwidth]{\e surveyBubblesNEFSCFall\D 1976\D 1980.pdf}}
                            \subfloat{\includegraphics[clip,trim={0 2.1cm 0.3cm 2.1cm},width=0.37\textwidth]{\e surveyBubblesNEFSCFall\D 1981\D 1985.pdf}}}{Figure 10}\\
                            \subfloat{\includegraphics[clip,trim={0 2.1cm 0.3cm 2.1cm},width=0.37\textwidth]{\e surveyBubblesNEFSCFall\D 1986\D 1990.pdf}}
                            \subfloat{\includegraphics[clip,trim={0 2.1cm 0.3cm 2.1cm},width=0.37\textwidth]{\e surveyBubblesNEFSCFall\D 1991\D 1995.pdf}}
                            \subfloat{\includegraphics[clip,trim={0 2.1cm 0.3cm 2.1cm},width=0.37\textwidth]{\e surveyBubblesNEFSCFall\D 1996\D 2000.pdf}}\\
                            \subfloat{\includegraphics[clip,trim={0 2.1cm 0.3cm 2.1cm},width=0.37\textwidth]{\e surveyBubblesNEFSCFall\D 2001\D 2005.pdf}}
                            \subfloat{\includegraphics[clip,trim={0 2.1cm 0.3cm 2.1cm},width=0.37\textwidth]{\e surveyBubblesNEFSCFall\D 2006\D 2010.pdf}}
                            \subfloat{\includegraphics[clip,trim={0 2.1cm 0.3cm 2.1cm},width=0.37\textwidth]{\e surveyBubblesNEFSCFall\D 2011\D 2015.pdf}}\\
                            \subfloat{\includegraphics[clip,trim={0 2.1cm 0.3cm 2.1cm},width=0.37\textwidth]{\e surveyBubblesNEFSCFall\D 2016\D 2016.pdf}}\\
         

         \caption{Map of the abundance of lobster captured during NEFSC's Fall Survey of the Gulf of Maine, Georges Bank and Scotian Shelf. Strata boundaries are outlined in red and LFA 41 stock boundaries are outlined in blue. Size of the symbols are scaled to the number of lobster observed within each tow. Black points represent tow locations with no lobsters.}
        \end{figure}
        \clearpage

%%%%Incidental catch

\begin{landscape}
        \begin{figure}
        \centering
                \pdftooltip{\subfloat{\includegraphics[width=1.3\textwidth]{\e Bycatch_Figure.pdf}}}{Figure 11}
        \caption{Estimated incidental catch rate (kg / t of lobsters) of fish species from the at-sea sampled data of the LFA 41 lobster fishery between 2006 to 2015 in 3 year time blocks.}
        \end{figure}
        \clearpage

\end{landscape}

%stratified analysis
%numbers
\begin{figure}
\centering
\pdftooltip{\subfloat{\includegraphics[width=0.5\textwidth]{\e lfa41DFOrestratifiednumbersNOY.png}}
\subfloat{\includegraphics[width=0.5\textwidth]{\e lfa41NEFSCSpringrestratifiednumbersNOY.png}}}{Figure 12}\\
\subfloat{\includegraphics[width=0.5\textwidth]{\e lfa41NEFSCFallrestratifiednumbersNOY.png}}
 \subfloat{\includegraphics[width=0.5\textwidth]{\e lfa41georgesnumbers.png}}

\caption{Stratified mean number per tow for the DFO RV Survey (RV 41 top left), NEFSC Spring Survey (NSpr41 top right), NEFSC Autumn Survey (NAut41 bottom left) and DFO Georges Bank Survey (GB bottom right) with surveys pruned to LFA 41. Within each plot the red line represents a three year running median. Confidence bounds are presented for each point estimate.}
\end{figure}
\clearpage




%dwao
\begin{figure}
\centering
\pdftooltip{\subfloat{\includegraphics[width=0.5\textwidth]{\e lfa41DFOrestratifiedDWAO.png}}
\subfloat{\includegraphics[width=0.5\textwidth]{\e lfa41NEFSCSpringrestratifiedDWAO.png}}}{Figure 13}\\
\subfloat{\includegraphics[width=0.5\textwidth]{\e lfa41NEFSCFallrestratifiedDWAO.png}}
    \includegraphics[width=.5\textwidth]{\e lfa41georgesDWAO.png}

\caption{Design weighted area occupied ($km^2$) of American lobster from DFO RV Survey (RV 41 top left), NEFSC Spring Survey (NSpr41 top right), NEFSC Autumn Survey (NAut41 bottom left) and DFO Georges Bank Survey (GB bottom right) with surveys pruned to LFA 41. Within each plot the red line represents a three year running median. }
\end{figure}
\clearpage



%gini
\begin{figure}
\centering
\pdftooltip{\subfloat{\includegraphics[width=0.5\textwidth]{\e lfa41DFOrestratifiedgini.png}}
\subfloat{\includegraphics[width=0.5\textwidth]{\e lfa41NEFSCSpringrestratifiedgini.png}}}{Figure 14}\\
\subfloat{\includegraphics[width=0.5\textwidth]{\e lfa41NEFSCFallrestratifiedgini.png}}
\subfloat{\includegraphics[width=0.5\textwidth]{\e lfa41georgesgini.png}}

\caption{Patchiness as estimated through the Gini index from DFO RV Survey (RV 41 top left), NEFSC Spring Survey (NSpr41 top right), NEFSC Autumn Survey (NAut41 bottom left) and DFO Georges Bank Survey (GB bottom right) with surveys pruned to LFA 41. Within each plot the red line represents a three year running median. Breaks in the three year running median are for years where no American lobster were captured in the survey strata.}
\end{figure}
\clearpage


%At Sea length freqs

\begin{figure}

\pdftooltip{\subfloat{\includegraphics[clip,trim={0 0.7cm 0cm 1.56cm},width=0.43\textwidth]{\e medianL\D SW\D Browns\D noseason.png}}
\subfloat{\includegraphics[clip,trim={0 0.7cm 0cm 1.56cm},width=0.43\textwidth]{\e medianL\D SE\D Browns\D noseason.png}}}{Figure 15}\\
\subfloat{\includegraphics[clip,trim={0 0.7cm 0cm 1.56cm},width=0.43\textwidth]{\e medianL\D Georges\D Basin\D noseason.png}}
\subfloat{\includegraphics[clip,trim={0 0.7cm 0cm 1.56cm},width=0.43\textwidth]{\e medianL\D Georges\D Bank\D noseason.png}}
                
    \caption{Median length (black line) with observed 25th and 75th quantiles (shaded polygon) from American lobster observed during at sampling of fishing activities. \emph{Upper}: Left - Southwest Browns; Right Southeast Browns; \emph{Lower}: Left - Georges Basin; Right - Georges Bank Summer. Within each plot red line represents a three year running median, whereas blue circles represent the annual sample sizes.}

\end{figure}



%max95 Length Observer data

\begin{figure}

\pdftooltip{\subfloat{\includegraphics[clip,trim={0 0.7cm 0cm 1.56cm},width=0.43\textwidth]{\e max95\D SW\D Browns\D noseason.pdf}}
\subfloat{\includegraphics[clip,trim={0 0.7cm 0cm 1.56cm},width=0.43\textwidth]{\e max95\D SE\D Browns\D noseason.pdf}}}{Figure 16}\\
\subfloat{\includegraphics[clip,trim={0 0.7cm 0cm 1.56cm},width=0.43\textwidth]{\e max95\D Georges\D Basin\D noseason.pdf}}
\subfloat{\includegraphics[clip,trim={0 0.7cm 0cm 1.56cm},width=0.43\textwidth]{\e max95\D Georges\D Bank\D noseason.pdf}}
                    
\caption{Maximum length (upper 95 quantile) of American lobster observed during at sampling of fishing activities. \emph{Upper}: Left - Southwest Browns ; Right Southeast Browns; \emph{Lower}: Left - Georges Basin; Right - Georges Bank . Within each plot red line represents a three year running median.}

\end{figure}


%Median Length 


\begin{figure}
\centering
\pdftooltip{\subfloat{\includegraphics[width=0.5\textwidth]{\e medianL\D LengthFrequenciesLFA41polygonSummerRV.png}}
\subfloat{\includegraphics[width=0.5\textwidth]{\e medianL\D LengthFrequenciesLFA41NEFSCspringrestratified.png}}}{Figure 17}\\
\subfloat{\includegraphics[width=0.5\textwidth]{\e medianL\D LengthFrequenciesLFA41NEFSCfallrestratified.png}}
\subfloat{\includegraphics[width=0.5\textwidth]{\e medianL\D LengthFrequenciesLFA41dfogeorges.png}}

\caption{Population weighted median carapace length (solid line and points) with accompanying first and third quartiles (shaded polygon) DFO RV Survey (RV 41 top left), NEFSC Spring Survey (NSpr41 top right), NEFSC Autumn Survey (NAut41 bottom left) and DFO Georges Bank Survey (GB bottom right) with surveys pruned to LFA 41. Within each plot the red line represents a three year running median. Breaks in the three year running median are for years where no American lobster were captured in the survey strata. Within each plot the blue points represent the annual sample sizes of observed lobster.}
\end{figure}
\clearpage


%Maximum Length 


\begin{figure}
\centering
\pdftooltip{\subfloat{\includegraphics[width=0.5\textwidth]{\e max95\D LengthFrequenciesLFA41polygonSummerRV.pdf}}
\subfloat{\includegraphics[width=0.5\textwidth]{\e max95\D LengthFrequenciesLFA41NEFSCspringrestratified.pdf}}}{Figure 18}\\
\subfloat{\includegraphics[width=0.5\textwidth]{\e max95\D LengthFrequenciesLFA41NEFSCfallrestratified.pdf}}
\subfloat{\includegraphics[width=0.5\textwidth]{\e max95\D LengthFrequenciesLFA41dfogeorges.pdf}}

\caption{Maximum carapace length (upper 95 quantile) of American lobster from DFO RV Survey (RV 41 top left), NEFSC Spring Survey (NSpr41 top right), NEFSC Autumn Survey (NAut41 bottom left) and DFO Georges Bank Survey (GB bottom rightr) with surveys pruned to LFA 41. Within each plot the red line represents a three year running median. Breaks in the three year running median are for years where no American lobster were captured in the survey strata. }
\end{figure}
\clearpage

%%Predator Index

\begin{figure}
\centering
    \pdftooltip{\includegraphics[width=.6\textwidth]{\e LobPredatorsbiomass.png}}{Figure 19}\\
    \includegraphics[width=.6\textwidth]{\e LobPredatorsabundance.png}\\
    \caption{Time series of biomass (lower) and abundance (upper) of predators of American lobster captured on the western Scotian Shelf during the summer RV survey.}

\end{figure}

%% bottom temperature

\begin{figure}

  \pdftooltip{  \includegraphics[width=.5\textwidth]{\e lfa41DFOTemp.png}
    \includegraphics[width=.5\textwidth]{\e lfa41NEFSCSpringTemps.png}}{Figure 20}\\
    \includegraphics[width=.5\textwidth]{\e lfa41NEFSCFallTemps.png}
    \includegraphics[width=.5\textwidth]{\e lfa41georgestemperature.png}\\
   
    \caption{Stratified mean temperatures from DFO RV summer (upper-left), NEFSC spring (upper-right), NEFSC fall (lower-left) and Georges Bank (lower-right), surveys with base strata for LFA 41. Within each plot red line represents running median and error bars are the 95 \% bootstrapped confidence intervals.}

\end{figure}



%species distribution modelling
\begin{landscape}
\begin{figure}
\centering

   \pdftooltip{ \subfloat{\includegraphics[clip,trim={0 3cm 0cm 3.5cm},width=.6\textwidth]{\e DepthSurface.png}}
    \subfloat{\includegraphics[clip,trim={0 3cm 0cm 3.5cm},width=.6\textwidth]{\e SlopeSurface.png}}}{Figure 21}\\
    \subfloat{\includegraphics[clip,trim={0 3cm 0cm 3.5cm},width=.6\textwidth]{\e curvature.png}}
    \caption{Interpolated surfaces for bathymetry, slope (log-scale) and curvature (log-scale) for the Scotian Shelf,Gulf of Maine and Georges Bank used as the projection layers for species distribution modeling. Planar coordinates are used for mapping with zone 20 specified.}
\end{figure}


\begin{figure}
\centering
   \pdftooltip{ \subfloat{\includegraphics[clip,trim={0 3cm 0cm 3.5cm},width=.33\textwidth]{\e temperature1970.png}}
    \subfloat{\includegraphics[clip,trim={0 3cm 0cm 3.5cm},width=.33\textwidth]{\e temperature1975.png}}
    \subfloat{\includegraphics[clip,trim={0 3cm 0cm 3.5cm},width=.33\textwidth]{\e temperature1980.png}}
    \subfloat{\includegraphics[clip,trim={0 3cm 0cm 3.5cm},width=.33\textwidth]{\e temperature1985.png}}}{Figure 22}\\
    \subfloat{\includegraphics[clip,trim={0 3cm 0cm 3.5cm},width=.33\textwidth]{\e temperature1990.png}}
    \subfloat{\includegraphics[clip,trim={0 3cm 0cm 3.5cm},width=.33\textwidth]{\e temperature1995.png}}
    \subfloat{\includegraphics[clip,trim={0 3cm 0cm 3.5cm},width=.33\textwidth]{\e temperature2000.png}}\\
    \subfloat{\includegraphics[clip,trim={0 3cm 0cm 3.5cm},width=.33\textwidth]{\e temperature2005.png}}
    \subfloat{\includegraphics[clip,trim={0 3cm 0cm 3.5cm},width=.33\textwidth]{\e temperature2010.png}}
    \subfloat{\includegraphics[clip,trim={0 3cm 0cm 3.5cm},width=.33\textwidth]{\e temperature2016.png}}

\caption{Interpolated temperature surfaces by year for the Scotian Shelf, Gulf of Maine and Georges Bank used as the projection layers for species distribution modeling. Planar coordinates are used for mapping with zone 20 specified.}
\end{figure}
\end{landscape}

\begin{figure}

   \pdftooltip{ \includegraphics[width=1\textwidth]{\e brtinfluenceplots.png}}{Figure 23}
    \caption{The relative influence of predictor variables Time (decimal year), temperature (t), depth (z), slope (dZ) and curvature (ddZ) from the boosted regression trees on the species distribution model.}

\end{figure}

\begin{landscape}
\begin{figure}
\centering

  \pdftooltip{  \subfloat{\includegraphics[clip,trim={0 1cm 0cm 1.5cm},width=.45\textwidth]{\e brtTime.png}}
    \subfloat{\includegraphics[clip,trim={0 1cm 0cm 1.5cm},width=.45\textwidth]{\e brtTemperature.png}}
    \subfloat{\includegraphics[clip,trim={0 1cm 0cm 1.5cm},width=.45\textwidth]{\e brtDepth.png}}}{Figure 24}\\
    \subfloat{\includegraphics[clip,trim={0 1cm 0cm 1.5cm},width=.45\textwidth]{\e brtCurvature.png}}
    \subfloat{\includegraphics[clip,trim={0 1cm 0cm 1.5cm},width=.45\textwidth]{\e brtSlope.png}}
    \caption{Fitted functions from the boosted regression tree models of lobster species distribution based on the variables of Time (decimal years), bottom temperature, depth, curvature and slope.}
\end{figure}




\begin{figure}
\centering
    \pdftooltip{\subfloat{\includegraphics[clip,trim={0 1.5cm 0cm 1.5cm},width=.33\textwidth]{\e trial\D continuousTimeboostedRegTree1970.png}}
    \subfloat{\includegraphics[clip,trim={0 1.5cm 0cm 1.5cm},width=.33\textwidth]{\e trial\D continuousTimeboostedRegTree1975.png}}
    \subfloat{\includegraphics[clip,trim={0 1.5cm 0cm 1.5cm},width=.33\textwidth]{\e trial\D continuousTimeboostedRegTree1980.png}}
    \subfloat{\includegraphics[clip,trim={0 1.5cm 0cm 1.5cm},width=.33\textwidth]{\e trial\D continuousTimeboostedRegTree1985.png}}}{Figure 25}\\
    \subfloat{\includegraphics[clip,trim={0 1.5cm 0cm 1.5cm},width=.33\textwidth]{\e trial\D continuousTimeboostedRegTree1990.png}}
    \subfloat{\includegraphics[clip,trim={0 1.5cm 0cm 1.5cm},width=.33\textwidth]{\e trial\D continuousTimeboostedRegTree1995.png}}
    \subfloat{\includegraphics[clip,trim={0 1.5cm 0cm 1.5cm},width=.33\textwidth]{\e trial\D continuousTimeboostedRegTree2000.png}}\\
    \subfloat{\includegraphics[clip,trim={0 1.5cm 0cm 1.5cm},width=.33\textwidth]{\e trial\D continuousTimeboostedRegTree2005.png}}
    \subfloat{\includegraphics[clip,trim={0 1.5cm 0cm 1.5cm},width=.33\textwidth]{\e trial\D continuousTimeboostedRegTree2010.png}}
    \subfloat{\includegraphics[clip,trim={0 1.5cm 0cm 1.5cm},width=.33\textwidth]{\e trial\D continuousTimeboostedRegTree2016.png}}
\caption{Predicted annual species distribution surfaces for American lobster from the boosted regression tree model results.  From left to right: top row - 1970, 1975, 1980, 1985; middle row - 1990, 1995, 2000; bottom row - 2005, 2010, 2016.}
\end{figure}

\end{landscape}

\begin{figure}

   \pdftooltip{ \includegraphics[width=1\textwidth]{\e ProportionSuitableLobsterHabitat35.png}}{Figure 26}
    \caption{The proportion of total area within LFA 41 representing $\geq 0.35$ probability of being suitable lobster habitat from the boosted regression tree results. }

\end{figure}


\begin{figure}

   \pdftooltip{ \includegraphics[width=1\textwidth]{\e AMO.png}}{Figure 27}
    \caption{Annual mean anomalies of the Atlantic multidecadal osscillation (AMO). Data obtained from \url{http://www.esrl.noaa.gov/psd/data/correlation//amon.us.long.data}}

\end{figure}

\begin{figure}

  \pdftooltip{  \includegraphics[width=1\textwidth]{\e CPUE.png}}{Figure 28}
    \caption{Catch per unit effort for lobster in the LFA 41 fishery. Y-axis labels were removed due to privacy concerns of the commercial catch rate levels.}

\end{figure}

%gini fishery
\begin{figure}

    \pdftooltip{\includegraphics[width=1\textwidth]{\e giniFootprintCPUE.png}}{Figure 29}
    \caption{Time series of spatial evenness of fishery catch rates (kg \textbackslash TH) estimated through the Gini Index for LFA 41. Red line represents the three year running median. Annual catch rates were estimated by grouping fishing trips into 0.05 deg$^2$.}

\end{figure}

\clearpage

%PCA
\begin{figure}

   \pdftooltip{ \includegraphics[width=0.5\textwidth]{\e indicators\D ReducedIndicatorsAssessment\D PC1.png}
    \includegraphics[width=0.5\textwidth]{\e indicators\D ReducedIndicatorsAssessment\D PC2.png}}{Figure 30}
    
    \caption{First and second axes of variation of the component scores from the ordination of the subset of biological and ecosystem indicators associated with the offshore LFA 41 lobster. Within each plot, the line represents a loess smoother through the component scores.}

\end{figure}
\clearpage



\begin{figure}

    \pdftooltip{\includegraphics[width=1\textwidth]{\e indicators\D ReducedIndicatorsAssessment\D anomalies.png}}{Figure 31}
    
    \caption{Time series of sorted ordination of the anomalies from the subset of biological and ecosystem indicators associated with LFA 41. Green indicates levels above the mean, whereas red indicates levels below the mean. White blocks indicate \textless 20 observations were available for that indicator and time period.}

\end{figure}



\begin{figure}
\centering
   \pdftooltip{ \includegraphics[width=.8\textwidth]{\e HCRExample.png}}{Figure 32}
    \caption{Example precautionary approach phase plot delimiting the healthy zone (green) above upper stock reference (USR) the cautious zone (yellow), between the USR and the limit reference point (LRP) and critical zone (red), below the LRP. The removal reference (RR) is shown as a solid black line in all three zones, however in practice the RR should be reduced in the cautious zone (black dashed) to allow stock rebuilding and set to 0 in the critical zone.
}

\end{figure}

%Biomass and Fishery reference points

%DFO Summer 
\begin{landscape}
\begin{figure}
\centering
       \pdftooltip{  \subfloat{\includegraphics[width=0.45\textwidth]{\e DFORefpointsNewArea.png}}
       \subfloat{\includegraphics[width=0.45\textwidth]{\e SpringRefpointsNewArea.png}}} {Figure 33}\\
        \subfloat{\includegraphics[width=0.45\textwidth]{\e AutumnRefpointsNewArea.png}}
        \subfloat{\includegraphics[width=0.45\textwidth]{\e GeorgesRefpointsNewArea.png}}\\
      
       \caption{Commercial biomass time series along with the running median (red line) the median of the five lowest non zero biomasses (\emph{LRI}; orange) and 40\% of the median of the higher productivity period (2000-2015;\emph{USI}, green). Top row: left - RV41, right - NSpr41. Bottom row: left - NAut41, right - GB}

\end{figure}
\end{landscape}

\begin{figure}
\centering
       \pdftooltip{ \subfloat{\includegraphics[width=0.45\textwidth]{\e relFDFOSurvey.png}}
       \subfloat{\includegraphics[width=0.45\textwidth]{\e relFSpringSurvey.png}}} {Figure 34}\\
       \subfloat{\includegraphics[width=0.45\textwidth]{\e relFAutumnSurvey.png}}
       \subfloat{\includegraphics[width=0.45\textwidth]{\e relFGeorgesSurvey.png}}\\

       \caption{Relative fishing mortality (relF) along with the running median (red line)for each survey. Top row: left - RV41, right - NSpr41. Bottom row: left - NAut41, right - GB }
\end{figure}
     \clearpage


\begin{landscape}
\begin{figure}
\centering
       \pdftooltip{      \subfloat{\includegraphics[width=0.45\textwidth]{\e HCRllbDataDFOSurvey.png}}
       \subfloat{\includegraphics[width=0.45\textwidth]{\e HCRllbDataSpringSurvey.png}}} {Figure 35}\\
        \subfloat{\includegraphics[width=0.45\textwidth]{\e HCRllbDataAutumnSurvey.png}}
        \subfloat{\includegraphics[width=0.45\textwidth]{\e HCRllbDataGeorgesSurvey.png}}\\

       
       \caption{Phase plots showing the relationship bewteen the running median of commercial biomasses and relative F for each survey. Top row: left - RV41, right - NSpr41. Bottom row: left - NAut41, right - GB.In all plots USI was defined using 40\% of the median of commercial biomass the higher productivity periods and LRI was defined as the median of the five lowest non zero biomasses. }

\end{figure}
\end{landscape}

\clearpage

%Reproducdtive potential refernce points

\begin{landscape}
\begin{figure}

\centering
   
      \pdftooltip{      
       \subfloat{\includegraphics[width=0.45\textwidth]{\e RefsRepPotDFO.png}}
       \subfloat{\includegraphics[width=0.45\textwidth]{\e RefsRepPotNEFSCSpring.png}}} {Figure 36}\\
       \subfloat{\includegraphics[width=0.45\textwidth]{\e RefsRepPotNEFSCAutumn.png}}
       \subfloat{\includegraphics[width=0.45\textwidth]{\e RefsRepPotGeorges.png}}\\

    
\caption{Reproductive potential in millions of eggs estimated from the four surveys covering LFA 41. Top row: left - RV41, right - NSpr41. Bottom row: left - NAut41, right - GB. Within panels reproductive potential time series along with the running median (red line). Where appropriate, the median of the five lowest non zero biomasses (lower boundary; orange) and  40\% of the median of the higher productivity period (upper boundary; green) are shown (see text for details). }
\end{figure}
\end{landscape}

\end{document}

